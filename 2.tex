\documentclass[a4paper]{D:/repositories/MyDGP/latex/PaperReadingLog}
\usepackage{caption}
\usepackage{overpic}
\usepackage{float}

\usepackage{enumitem} 
\usepackage{amsfonts,amssymb,amsmath}
%\usepackage[linesnumbered,ruled,lined]{algorithm2e}%%伪代码
\usepackage{multicol}
\usepackage{wrapfig}
\usepackage{algorithm}%%算法伪代码,支持中文
\usepackage{algorithmic}

\setmainfont{Arial}

\begin{document}

\PaperInfo
{拟牛顿法}
{朱}
{天宇}
{}

牛顿法需要求函数的Hessian矩阵,对于大规模的问题,这需要很大的代价。\textbf{拟牛顿法}使用Hessian矩阵或者其逆矩阵来近似,来替代求解Hessian矩阵的过程。其近似的矩阵记为$\mathrm{B^k}$,保留了Hessian矩阵的部分性质。

\section{割线方程}
牛顿法的推导基于函数$f(x)$的泰勒二次展开:
$$
\nabla f(x)=\nabla f(x^{k+1})+\nabla^2f(x^{k+1})(x-x^{k+1})+\mathcal{O}(\lVert x-x^{k+1}\lVert^2)
$$

这里,令$x=x^k,s^k=x^{k+1}-x^k,y^k=\nabla f(x^{k+1})-\nabla f(x^k)$,上式可以写成
$$
\nabla^2f(x^{k+1})s^k+\mathcal{O}(\lVert s^k \lVert^2)=y^k
$$
忽略高阶项$\mathcal{O}$,那么我们希望Hessian的近似矩阵$\mathrm{B}^{k+1}$满足
$$
y^k=\mathrm{B}^{k+1}s^k
$$
使用$\mathrm{H}^{k+1}$表示$\mathrm{B}^k$的逆矩阵,就有
$$
s^k=\mathrm{H}^{k+1}y^k
$$
这两个式子被称为\textbf{割线方程}。

割线方程的本质是希望目标函数$f(x)$和其二次近似$m_{k+1}(d)=f(x^{k+1})+\nabla f(x^{k+1})^\top d+\frac{1}{2}d^\top \mathrm{B}^{k+1}d$,在$x=x^k,x=x^k+1$处有相同的梯度。

\paragraph{曲率条件}
近似矩阵$\mathrm{B}^k$需要正定,对式子$y^k=\mathrm{B}^{k+1}s^k$两边同时左乘$(s^k)^\top$,就可以得到\textbf{必要条件}
$$
(s^k)^\top \mathrm{B}^{k+1}s^k=(s^k)^\top y^k>0
$$
这被称为曲率条件。

在\textbf{线搜索}的时候,需要使用Wolfe准则来使得曲率条件成立。根据Wolfe准则的第二个条件$\varphi'(\alpha)\ge \sigma\varphi'(0)$,写成梯度形式,两侧同时乘以$s^k$,就有
$$
\nabla f(x^{k+1})^\top s^k\ge \sigma\nabla f(x^k)^\top s^k
$$
两边同时减去$\nabla f(x^k)^\top s^k$,就有
$$
(y^k)^\top s^k\ge (\sigma-1)\nabla f(x^k)^\top s^k>0
$$
因为$\sigma<1$,且$s^k=\alpha_kd^k$,是下降方向。

通常,近似矩阵$\mathrm{B}^k,\mathrm{H}^k$都迭代更新的。有的方法近似$\mathrm{B}^k$,有的方法近似$\mathrm{H}^k$

\section{秩一更新SR1}
使用\textbf{待定系数法},已知$\mathrm{B}^k$,满足割线方程,求$\mathrm{B}^{k+1}$。

设$\mathrm{B}^{k+1}=\mathrm{B}^k+auu^\top$,带入割线方程
$$
\mathrm{B}^{k+1}s^k=(\mathrm{B}^k+auu^\top)s^k=y^k
$$
从而有
$$
(a\cdot u^\top s^k)u=y^k-\mathrm{B}^ks^k
$$
由于$(a\cdot u^\top s^k)$是标量,因此$u$与$y^k-\mathrm{B}^ks^k$方向是相同的。不妨令$u=y^k-\mathrm{B}^ks^k$,带入上式,有
$$
a((y^k-\mathrm{B}^ks^k)^\top s^k)(y^k-\mathrm{B}^ks^k)=y^k-\mathrm{B}^ks^k
$$
可得$a=\frac{1}{(y^k-\mathrm{B}^ks^k)^\top s^k}$,带入最初的式子,就有
$$
\mathrm{B}^{k+1}=\mathrm{B}^k+\frac{(y^k-\mathrm{B}^ks^k)(y^k-\mathrm{B}^ks^k)^\top}{(y^k-\mathrm{B}^ks^k)^\top s^k}
$$
相同方法可得
$$
\mathrm{H}^{k+1}=\mathrm{H}^k+\frac{(s^k-\mathrm{H}^ky^k)(s^k-\mathrm{H}^ky^k)^\top}{(s^k-\mathrm{H}^ky^k)^\top s^k}
$$

一个有趣的观察是,将公式1做如下替换则可获得公式2:
$$
\mathrm{B}^k\rightarrow\mathrm{H}^k,\quad s^k\rightarrow y^k
$$

SR1公式结构简单,但无法保证矩阵在迭代中保持正定。之前的条件只是充分条件。

\section{秩二更新BFGS}

同样使用待定系数法,设
$$
\mathrm{B}^{k+1}=\mathrm{B}^k+auu^\top+bvv^\top
$$

可推出基于$\mathrm{B}^k$的更新公式:
$$
\mathrm{B}^{k+1}=\mathrm{B}^k+\frac{y^k(y^k)^\top}{(s^k)^\top y^k}-\frac{\mathrm{B}^ks^k(\mathrm{B}^ks^k)^\top}{(s^k)^\top \mathrm{B}^ks^k}
$$

带入SMW公式可以获得$\mathrm{H}^k$的更新公式:
$$
\mathrm{H}^{k+1}=(\mathrm{I}-\rho_ks^k(y^k)^\top)^\top\mathrm{H}^k(\mathrm{I}-\rho_ks^k(y^k)^\top)+\rho_ks^k(s^k)^\top
$$
其中,$\rho_k=\frac{1}{(s^k)^\top y^k}$

BFGS公式产生的矩阵$\mathrm{H}^{k+1}$正定的条件是满足不等式$(s^k)^\top y^k>0$,这可以通过线搜索保证。

\section{有限内存的BFGS}
大规模问题需要存储拟牛顿矩阵$\mathrm{B}^k$或者$\mathrm{H}^k$,需要消耗$\mathcal{O}(n^2)$的内存,不现实。LBFGS可以解决这个问题。

首先,基于$\mathrm{H}^k$的BFGS,引入新符号后的表示为
$$
\mathrm{H}^{k+1}=(\mathrm{V}^k)^\top \mathrm{H}^k\mathrm{V}^k+\rho_ks^k(s^k)^\top
$$
其中,$\rho_k=\frac{1}{(s^k)^\top y^k}, \mathrm{V}^k=\mathrm{I}-\rho_ky^k(s^k)^\top$

这个公式是一个\key{类似递推的形式},因此尝试展开$m$次:
$$
\begin{aligned}
   &\mathrm{H}^k\\
   &=(\mathrm{V}^{k-m}...\mathrm{V}^{k-1})^\top \mathrm{H}^{k-m}(\mathrm{V}^{k-m}...\mathrm{V}^{k-1})+\\
   &\rho_{k-m}(\mathrm{V}^{k-m+1}...\mathrm{V}^{k-1})^\top s^{k-m}(s^{k-m})^\top (\mathrm{V}^{k-m+1}...\mathrm{V}^{k-1})+\\
   &\rho_{k-m+1}(\mathrm{V}^{k-m+2}...\mathrm{V}^{k-1})^\top s^{k-m+1}(s^{k-m+1})^\top (\mathrm{V}^{k-m+2}...\mathrm{V}^{k-1})+\\
   &...+\\
   &\rho_{k-1}s^{k-1}(s^{k-1})^\top
\end{aligned}
$$

这样,通过$\mathrm{H}^{k-m}$就可以获得$\mathrm{H}^k$。但$\mathrm{H}^{k-m}$无法显示求出,所以要找\textbf{近似矩阵}$\hat{\mathrm{H}}^{k-m}$,一般来说这个矩阵的结构要比较简单。

实际上,$\mathrm{H}^k$的显式形式根本无需计算。只要计算$\mathrm{H}^k\nabla f(x^k)$即可。可以通过这个算法巧妙地计算$\mathrm{H}^k\nabla f(x^k)$。

\begin{algorithm}[H]
	\caption{L-BFGS双循环递归算法} 
	%\label{alg:alg1}
	\begin{algorithmic}
		% 输入
		\REQUIRE $n \geq 0 \vee x \neq 0$ 
		% 输出
		\ENSURE $y = x^n$  较看脸
		
		% 初始化
		\STATE $y \leftarrow 1$ 
		
		% 逻辑
		\IF{$n < 0$} 
			\STATE $X \leftarrow 1 / x$ 
			\STATE $N \leftarrow -n$  jlk就
		\ELSE 
			\STATE $X \leftarrow x$ 
			\STATE $N \leftarrow n$
		\ENDIF
		
		\WHILE{$N \neq 0$} 
			\IF{$N$ is even} 
				\STATE $X \leftarrow X \times X$ 
				\STATE $N \leftarrow N / 2$ 
			\ELSIF{$N$ is odd}
				\STATE $y \leftarrow y \times X$ 
				\STATE $N \leftarrow N - 1$ 
			\ENDIF 
		\ENDWHILE
	\end{algorithmic}
\end{algorithm}

\end{document}
