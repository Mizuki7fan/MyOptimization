\documentclass[a4paper]{D:/MyRepo/Script/latex/PaperReadingLog}
\usepackage{caption}
\usepackage{overpic}
\usepackage{float}

\usepackage{enumitem} 
\usepackage{amsfonts,amssymb,amsmath}

\usepackage{algorithm}%%算法伪代码,支持中文
\usepackage{algorithmic}
\setmainfont{Arial}

\begin{document}

\PaperInfo
{7.凸优化算法}
{朱}
{天宇}
{}

如果要优化的函数是凸的,一般采用\key{下降算法}。

\section{等式约束}

\subsection{两条路径}

\begin{equation}
    \label{equ1}
    \begin{aligned}
        \min\quad &f(\mathrm{x})\\
        s.t.\quad &\mathrm{A}\mathrm{x}=\mathrm{b}
    \end{aligned}
\end{equation}

这里$f(\mathrm{x})$是凸的,矩阵$\mathrm{A}\in\mathbb{R}^{p\times n},\mathrm{rankA}=p<n$。

其KKT系统为:
\begin{equation}
    \label{equ3}
    \left\{
        \begin{aligned}
            \mathrm{A}\mathrm{x}^\star-\mathrm{b}=0\\
            \nabla f(\mathrm{x}^\star)+\mathrm{A}^\top\nu^\star=0
        \end{aligned}
        \right.        
\end{equation}

这里$\nu$是拉格朗日乘子。由于没有不等式约束,所以KKT系统中的1/3/4式始终满足。剩余的是2/5式。上面的式子被称为原问题可行性方程,下面的式子被称为对偶问题可行性方程。这个系统关于$\nu$是线性的,关于$\mathrm{x}^\star$是非线性的,因此KKT系统是非线性的。

求解这个系统,一般通过牛顿法来处理。原问题$\min f(\mathrm{x})$的牛顿法和求解KKT系统的牛顿法是差不多的。原问题的牛顿法通常是在$f(\mathrm{x})$处做2阶泰勒展开,变成二次函数,获得当前点处的牛顿方向。展开后的二次函数的导数显然是线性的;而对于KKT非线性系统的牛顿迭代,实际上是做一阶近似。对于原问题的二阶近似,会得到一个梯度和Hessian;对于KKT非线性系统的一阶近似,会得到一个Jacobian矩阵。

为了确定问题\ref{equ1}的牛顿步,在$\mathrm{x}$附近作二阶泰勒展开,得到一个关于$\mathrm{s}$的式子:
\begin{equation}
    \label{equ2}
    \begin{aligned}
        \min\quad&\hat{f}(\mathrm{x}+\mathrm{s})=f(\mathrm{x})+\nabla f(\mathrm{x})^\top +\frac{1}{2}\mathrm{s}^\top \nabla^2f(\mathrm{x})\mathrm{s}\\
        s.t.\quad&\mathrm{A}(\mathrm{x}+\mathrm{s})=\mathrm{b}
    \end{aligned}        
\end{equation}

对于这个式子,假设最优解是$\delta_{\mathrm{x}_{nt}}$。则通过\ref{equ2}的KKT条件,就可以知道存在关联的最优对偶变量$w\in\mathbb{R}^p$,满足
\begin{equation}
    \label{equ5}
    \left[
\begin{array}[]{cc}
    \nabla^2 f(\mathrm{x}) & \mathrm{A}^\top \\
    \mathrm{A}^\top & 0
\end{array}
\right]
\left[
\begin{array}[]{c}
    \delta_{\mathrm{x}_{nt}}\\
    w
\end{array}
\right]
=
\left[
\begin{array}[]{c}
    -\nabla f(\mathrm{x})\\
    0
\end{array}
\right]
\end{equation}


从而可以解出牛顿步长。

而如果不对原问题\ref{equ1}进行二阶泰勒展开,而是直接处理其KKT条件,替换\ref{equ3}中的$\mathrm{x}^\star,\nu^\star$为$\mathrm{x}+\delta_{\mathrm{x}_{nt}},w$,并且将第二个式子中的梯度项替换为$\mathrm{x}$附近的线性近似,就可以获得
\begin{equation}
    \label{equ4}
    \left\{
    \begin{aligned}
        &\mathrm{A}(\mathrm{x}+\delta_{\mathrm{x}_{nt}})-\mathrm{b}=0\\
        &\nabla f(\mathrm{x}+\delta_{\mathrm{x}_{nt}})+\mathrm{A}^\top w\approx \nabla f(\mathrm{x})+\nabla^2 f(\mathrm{x})\delta_{\mathrm{x}_{nt}}+\mathrm{A}^\top w=0
    \end{aligned}
    \right.
\end{equation}

可以发现\ref{equ4}与\ref{equ5}是一模一样的:

原问题\ref{equ1}$\stackrel{\text{二阶泰勒展开}}{\Rightarrow}$二次规划(QP)问题\ref{equ2}$\Rightarrow$问题\ref{equ2}的KKT\ref{equ5}$\Rightarrow$牛顿迭代步$\delta$

原问题\ref{equ1}$\Rightarrow$原问题\ref{equ1}的KKT\ref{equ3}$\stackrel{\text{一阶近似}}{\Rightarrow}$\ref{equ3}的一阶近似展开$\Rightarrow$牛顿迭代步$\delta$

这两条路径中,第一条路经可以用于无约束问题;

\subsection{停机准则}
定义\textbf{下降量}$\kappa$
$$
\kappa(\mathrm{x})=\sqrt{\delta_{\mathrm{x}_{nt}^\top} \nabla^2f(\mathrm{x})\delta_{\mathrm{x}_{nt}}}
$$

因为这个值等于
$$
\frac{\mathrm{d}}{\mathrm{d}\alpha} f(\mathrm{x}+\alpha\delta_{\mathrm{x}_{nt}})\left.\right|_{\alpha=0}=\nabla f(\mathrm{x})^\top \delta_{\mathrm{x}_{nt}}=-\kappa(\mathrm{x})^2
$$

如果$\kappa(\mathrm{x})$比较小,则算法停止。

\subsection{算法流程(初始点可行的时候)}

\begin{algorithm}[H]
	\caption{初始点为可行点的等式约束凸问题的牛顿迭代方法} 
	\begin{algorithmic}[1]
		% 初始化
		\STATE 初始点为$\mathrm{x}\in\mathrm{dom} f$,满足$\mathrm{A}\mathrm{x}=\mathrm{b}$,定义tolerance$\epsilon>0$
        \WHILE{1}
        \STATE \text{计算牛顿步、下降方向$\delta_{\mathrm{x}_{nt}}$和下降量}$\kappa(\mathrm{x})$
        \IF{$\kappa^2/2\le\epsilon$}
            \STATE \text{算法终止}
        \ENDIF
        \STATE \text{通过backtarcking line search以确定步长}$\alpha$
        \STATE \text{更新:}$\mathrm{x}\leftarrow \mathrm{x}+\alpha\delta_{\mathrm{x}_{nt}}$
        \ENDWHILE
	\end{algorithmic}
\end{algorithm}

值得注意的是,这个过程中,初始点必须是可行的。如果起始点不一定可行,要用外点法相关的方法。

\subsection{起始点不可行的情况}
当$\mathrm{x}$是可行点的时候,其天然满足$\mathrm{A}\mathrm{x}=\mathrm{b}$的约束。起始点是不可行点的时候,不能假定这个条件满足。

这时候,就要寻找一个$\delta_{\mathrm{x}}$使得$\mathrm{A}(\mathrm{x}+\delta_{\mathrm{x}})=\mathrm{b}$。

记录满足约束的点为$\mathrm{x}^\star=\mathrm{x}+\delta_\mathrm{x},\mathrm{A}\mathrm{x}^\star=\mathrm{b}$,同样将原问题\ref{equ1}的KKT系统\ref{equ3}中的$\mathrm{x}^\star$替换掉,并且进行一阶展开,就得到
\begin{equation}
    \label{equ6}
    \left\{
    \begin{aligned}
        &\mathrm{A}(\mathrm{x}+\delta_\mathrm{x})-\mathrm{b}=0\\
        &\nabla f(\mathrm{x}+\delta_\mathrm{x})+\mathrm{A}^\top \mu\approx \nabla f(\mathrm{x})+\nabla^2 f(\mathrm{x})\delta_\mathrm{x}+\mathrm{A}^\top \mu=0
    \end{aligned}
    \right.
\end{equation}

同样可以得到一个KKT系统:
\begin{equation}
    \label{equ7}
    \left[
\begin{array}[]{cc}
    \nabla^2 f(\mathrm{x}) & \mathrm{A}^\top \\
    \mathrm{A}^\top & 0
\end{array}
\right]
\left[
\begin{array}[]{c}
    \delta_\mathrm{x}\\
    \mu
\end{array}
\right]
=
\left[
\begin{array}[]{c}
    -\nabla f(\mathrm{x})\\
    \mathrm{A}\mathrm{x}-\mathrm{b}
\end{array}
\right]
\end{equation}

注意方程组\ref{equ7}和方程组\ref{equ5}的区别:等号右边下方,\ref{equ5}是0,而\ref{equ7}是$\mathrm{A}\mathrm{x}-\mathrm{b}$。

也就是说,当当前点是不可行点的时候,迭代时只需要将这一项进行修改就行了。

\paragraph{解释}
将KKT系统表达为函数$r(\mathrm{x}^\star,\nu^\star)=0,r:\mathbb{R}^n\times\mathbb{R}^p\mapsto\mathbb{R}^n\times\mathbb{R}^p$,其包含两个分部:primal和dual的。
$$
r(\mathrm{x}^\star,\nu^\star)=(r_{dual}(\mathrm{x}^\star,\nu^\star),r_{primal}(\mathrm{x}^\star,\nu^\star))
$$

在这里,$r_{dual},r_{pri}$分别称为对偶残差和原残差。
$$
\begin{aligned}
    &r_{dual}(\mathrm{x},\nu)=\nabla f(\mathrm{x})+\mathrm{A}^\top \nu\\
    &r_{primal}(\mathrm{x},\nu)=\mathrm{A}\mathrm{x}-\mathrm{b}
\end{aligned}
$$

在当前点$\mathrm{y}=(\mathrm{x},\nu)$的附近,作$r$的一阶近似展开,就有
$$
r(y+\delta_y)\approx \hat{r}(y+\delta_y)=r(y)+\mathrm{J}[r(y)]\delta_y
$$

这里,$\mathrm{J}$是Jacobian矩阵,因为KKT条件当中已经存在一阶信息了,再次作一阶展开,就会出现Jacobian矩阵。表示在$y$点处对$r$进行求导。$\mathrm{J}[r(y)]$的维度是$(n+p)\times (n+p)$。

定义$\delta_{y_{pd}}$为$\hat{r}(y+\delta_y)=0$的时候的牛顿迭代步,由于$\hat{r}=0$,可知
$$
\mathrm{J}[r(y)]\delta_{y_{pd}}=-r(y)
$$

注意到,$\delta_{y_{pd}}=(\delta_{x_{pd}},\delta_{y_{pd}})$,同时给出了primal和dual的步长。

改写一下\ref{equ7}式:
\begin{equation}
    \label{equ8}
    \left[
\begin{array}[]{cc}
    \nabla^2 f(\mathrm{x}) & \mathrm{A}^\top \\
    \mathrm{A}^\top & 0
\end{array}
\right]
\left[
\begin{array}[]{c}
    \delta_\mathrm{x_{pd}}\\
    \delta_{\nu_{pd}}
\end{array}
\right]
=-
\left[
\begin{array}[]{c}
  r_{dual}\\
  r_{pri}  
\end{array}
\right]
=-
\left[
\begin{array}[]{c}
    \nabla f(\mathrm{x})+\mathrm{A}^\top\nu\\
    \mathrm{A}\mathrm{x}-\mathrm{b}
\end{array}
\right]
\end{equation}

这里,$\nu+\delta_{\nu_{pd}}=\mu$

\subsection{算法流程(初始点不可行的时候)}
\begin{algorithm}[H]
	\caption{初始点为不可行点的等式约束凸问题的牛顿迭代方法} 
	\begin{algorithmic}[1]
		% 初始化
		\STATE 初始点为$\mathrm{x}\in\mathrm{dom} f$,但是$\mathrm{A}\mathrm{x}\neq\mathrm{b}$,定义tolerance$\epsilon>0$,定义$\tau\in(0,1/2),\gamma\in(0,1)$
        \WHILE{$\mathrm{A}\mathrm{x}\neq \mathrm{b} \text{或者} \lVert r(\mathrm{x},\nu)\lVert_2>\epsilon$}
            \STATE \text{计算primal和dual的牛顿步$\delta_{\mathrm{x}_{nt}},\delta_{\nu_{nt}}$}
            \STATE \text{在$\lVert r \lVert_2$上做backtracking line search}
            \STATE $\alpha\leftarrow 1$
         \WHILE{$\lVert r(\mathrm{x}+\alpha\delta_{\mathrm{x}_{nt}},\nu+\alpha\delta_{\nu_{nt}})\lVert_2>(1-\tau\alpha)\lVert r(\mathrm{x},\nu)\lVert_2$}
             \STATE $\alpha\leftarrow\gamma\alpha$
        \ENDWHILE
        \STATE $\mathrm{x}\leftarrow\mathrm{x}+\alpha\delta_{\mathrm{x}_{nt}}$
        \STATE $\nu\leftarrow\nu+\alpha\delta_{\nu_{nt}}$
        \ENDWHILE
	\end{algorithmic}
\end{algorithm}

\section{不等式约束}
含不等式约束的凸优化问题的形式为:
\begin{equation}
    \label{equ9}
    \begin{aligned}
        \min\quad &f_0(\mathrm{x})\\
        s.t.\quad &f_i(\mathrm{x})\le0,\quad 1,...,m\\
        &\mathrm{A}\mathrm{x}=\mathrm{b}
    \end{aligned}
\end{equation}

其KKT系统为
\begin{equation}
    \label{equ10}
    \left\{
    \begin{aligned}
        f_i(\mathrm{x}^\star)& \le0,i=1,...,m\\
        \mathrm{A}\mathrm{x}^\star&=b\\
        \lambda_i^\star&\ge0\\
        \lambda_i^\star f_i(\mathrm{x}^\star)&=0,i=1,...,m\\
        \nabla f_0(\mathrm{x}^\star)+\sum_{i=1}^m\lambda_i^\star \nabla f_i(\mathrm{x}^\star)+\mathrm{A}^\top\nu^\star&=0
    \end{aligned}
    \right.
\end{equation}

这个系统含有不等式约束,无法像之前一样处理。对于这种情况,有2种处理方法:\key{障碍函数}、\key{primal-dual 内点算法}

\subsection{使用障碍函数}
定义\key{示性函数}:
$$
I_-(u)=\left\{
    \begin{aligned}
        0,\quad u\le0\\
        \infty,\quad u>0
    \end{aligned}
\right.
$$

通过示性函数,可以将原问题改写称为仅含有等式约束的问题:
\begin{equation}
    \label{equ11}
    \begin{aligned}
        \min\quad&f_0(\mathrm{x})+\sum_{i=1}^mI_-(f_i(\mathrm{x}))\\
        s.t.\quad&\mathrm{A}\mathrm{x}=\mathrm{b}
    \end{aligned}
\end{equation}

示性函数不可微不连续,性质不好,所以引入\key{障碍函数}来进行\textbf{逼近}。比如使用
$$
\hat{I}_-(u)=-(1/t)\log(-u)
$$

这张图显示了障碍函数和示性函数的关系:t越大,近似程度越高。
\begin{figure}[H]%
    \centering
    \begin{overpic}[width=0.80\linewidth]{fig/7.1.png}
    \end{overpic}
    \vspace{-3.5mm}
    \vspace{2mm}
\end{figure}

代入障碍函数后,方程\ref{equ11}就变为:
\begin{equation}
    \label{equ12}
    \begin{aligned}
        \min\quad&f_0(\mathrm{x})+\sum_{i=1}^m-(1/t)\log(-f_i(\mathrm{x}))\\
        s.t.\quad&\mathrm{A}\mathrm{x}=\mathrm{b}
    \end{aligned}
\end{equation}

这是一个等式约束的凸优化问题。

这里,障碍函数被称为\key{对数障碍函数},其表达式、一阶导、二阶导分别是:
\begin{equation}
    \label{equ13}
    \begin{aligned}
        \phi(\mathrm{x})&=-\sum_{i=1}^m\log(-f_i(\mathrm{x}))\\
        \nabla \phi(\mathrm{x})&=\sum_{i=1}^m\frac{1}{-f_i(\mathrm{x})}\nabla f_i(\mathrm{x})\\
        \nabla^2 \phi(\mathrm{x})&=\sum_{i=1}^m\frac{1}{f_i(\mathrm{x})^2}\nabla f_i(\mathrm{x})\nabla f_i(\mathrm{x})^\top+\sum_{i=1}^m\frac{1}{-f_i(\mathrm{x})}\nabla^2 f_i(\mathrm{x})
    \end{aligned}
\end{equation}

方程\ref{equ12}可以通过牛顿法来求解。可以将目标函数乘以$t$,则变为
\begin{equation}
    \label{equ14}
    \begin{aligned}
        \min\quad&tf_0(\mathrm{x})+\phi(\mathrm{x})\\
        s.t.\quad&\mathrm{A}\mathrm{x}=\mathrm{b}
    \end{aligned}
\end{equation}

\paragraph{中心路经central path}
对于每一个$t$,都能对应到一个比较好的$\mathrm{x}$,因此,定义$\mathrm{x}^\star(t)=\arg\min_\mathrm{x}\{tf_0(\mathrm{x})+\phi(\mathrm{x})\ s.t.\ \mathrm{A}\mathrm{x}=\mathrm{b}\}$。实际算法中需要仔细调节$t$的数值。

$\mathrm{x}^\star(t)$和原问题\ref{equ9}不是一回事,是问题\ref{equ14}的最优解。问题\ref{equ9}的\key{中心路径},被定义为由$\mathrm{x}^\star(t)$构成的,按照t的值进行排序的有序点集合$\{\mathrm{x}^\star(t)|t>0\}$。另外值得注意的是,当$t\rightarrow\infty$的时候,障碍项趋向于不起作用。

在中心路径上的点,始终保证$\mathrm{x}^\star(t)$是可行点。也就是其满足约束
$$
\mathrm{A}\mathrm{x}^\star(t)=\mathrm{b},\quad f_i(\mathrm{x}^\star(t))<0,\quad i=1,...,m
$$

并且$\exists\hat{\nu}\in\mathbb{R}^p$,满足问题\ref{equ14}的\textbf{最优性条件}:
\begin{equation}
    \label{equ15}
    \begin{aligned}
        0&=t\nabla f_0(\mathrm{x}^\star(t))+\nabla\phi(\mathrm{x}^\star(t))+\mathrm{A}^\top\hat{\nu}\\
        &=t\nabla f_0(\mathrm{x}^\star(t))+\sum_{i=1}^m\frac{1}{-f_i(\mathrm{x}^\star(t))}\nabla f_i(\mathrm{x}^\star(t))+\mathrm{A}^\top\hat{\nu}
    \end{aligned}        
\end{equation}

定义$\lambda^\star(t)=-\frac{1}{tf_i(\mathrm{x}^\star(t))}>0,\nu^\star(t)=\frac{\hat{\nu}}{t}$,将之带入式子\ref{equ15}中,可以得到
\begin{equation}
    \label{equ16}
    \nabla f_0(\mathrm{x}^\star(t))+\sum_{i=1}^m\lambda_i^\star(t)\nabla f_i(\mathrm{x}^\star(t))+\mathrm{A}^\top\nu^\star(t)=0
\end{equation}

可以发现这个式子和原问题\ref{equ9}的KKT条件\ref{equ10}的最后一项是一样的。区别仅仅在于有无(t)。而这个式子\ref{equ16}是原问题的\ref{equ9}的拉格朗日函数
$$
L(\mathrm{x},\lambda,\nu)=f_0(\mathrm{x})+\sum_{i=1}^m\lambda_if_i(\mathrm{x})+\nu^\top(\mathrm{A}\mathrm{x}-\mathrm{b})
$$
的导数。因此,$(\lambda^\star(t),\nu^\star(t))$是一对可行的对偶乘子。

所以可以讨论原问题\ref{equ14}的\textbf{对偶函数}
$$
\begin{aligned}
    g(\lambda^\star(t),\nu^\star(t))&=\min_\mathrm{x}L(\mathrm{x},\lambda^\star(t),\nu^\star(t))\\
    &=f_0(\mathrm{x}^\star(t))+\sum_{i=1}^m\lambda^\star_i(t)f_i(\mathrm{x}^\star(t))+\nu^\star(t)^\top(\mathrm{A}\mathrm{x}^\star(t)-\mathrm{b})\\
    &=f_0(\mathrm{x}^\star(t))-m/t
\end{aligned}
$$

关于最后一行的化简。回顾定义$\lambda^\star(t)=-\frac{1}{tf_i(\mathrm{x}^\star(t))}$以及$\mathrm{A}\mathrm{x}^\star(t)-\mathrm{b}=0$。

发现原问题和对偶问题之间存在一个值为$m/t$的gap。因此,$f_0(\mathrm{x}^\star(t))-v^\star\le m/t$。这样可以说明$t\rightarrow\infty$的时候,$\mathrm{x}^\star(t)\rightarrow\mathrm{x}^\star$。

\paragraph{从KKT出发的理解}
假设$\mathrm{x}$是问题\ref{equ14}的唯一解。那么就必然$\exists \lambda,\nu$使得:
\begin{equation}
    \label{equ17}
    \left\{
    \begin{aligned}
        f_i(\mathrm{x})& \le0,i=1,...,m\\
        \mathrm{A}\mathrm{x}&=b\\
        \lambda_i&\ge0\\
        -\lambda_i f_i(\mathrm{x})&=1/t,i=1,...,m\\
        \nabla f_0(\mathrm{x})+\sum_{i=1}^m\lambda_i \nabla f_i(\mathrm{x})+\mathrm{A}^\top\nu&=0
    \end{aligned}
    \right.
\end{equation}

和\ref{equ10}的KKT系统唯一的区别就是互补充实条件$\lambda_if_i(\mathrm{x})$等号右边的值从0变成了$1/t$,对于一个较大的$t$,$\mathrm{x}^\star(t),\lambda^\star(t),\nu^\star(t)$接近于KKT条件中的最优值。


\subsubsection{障碍函数法算法流程}
\begin{algorithm}
	\caption{障碍函数法} 
	\begin{algorithmic}[1]
		% 初始化
		\STATE 选择可行点$\mathrm{x},t\leftarrow t_0>0,\gamma>1,\epsilon>0$
        \WHILE{1}
        \STATE 通过对方程\ref{equ14}求最小化,获取$\mathrm{x}^\star(t)$
        \STATE $\mathrm{x}\leftarrow\mathrm{x}^\star(t)$
        \IF{$m/t<\epsilon$}
            \STATE \text{算法终止}
        \ENDIF
        \STATE 增加$t$的值:$t\leftarrow \gamma t$
        \ENDWHILE
	\end{algorithmic}
\end{algorithm}

获取$\mathrm{x}^\star(t)$的步骤被称为\key{外部迭代},$\mathrm{x}^\star(t)$无需精确的计算;选择$\gamma$是一个trade-off:外部迭代和内部迭代的次数。$t_0$的选择也是一个trade-off:如果$t_0$过大,那么首次外部迭代需要多轮;$t_0$过小,则算法需要额外的外部迭代;

\subsection{直接修改KKT系统}



\end{document}