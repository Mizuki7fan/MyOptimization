\documentclass[a4paper]{D:/MyRepo/Script/latex/PaperReadingLog}
\usepackage{caption}
\usepackage{overpic}
\usepackage{float}

\usepackage{enumitem} 
\usepackage{amsfonts,amssymb,amsmath}

\usepackage{algorithm}%%算法伪代码,支持中文
\usepackage{algorithmic}
\setmainfont{Arial}

\begin{document}

\PaperInfo
{5.凸优化问题的定义}
{朱}
{天宇}
{}

\section{凸的定义}
\subsection{仿射集Affine set}
如果一个集合的任意两个元素的线性组合都在这个集合内,称这个集合为仿射集\footnote{https://blog.csdn.net/robert\_chen1988/article/details/80048245},即对于集合C,有:
$$
\forall x,y\in C,\theta\in\mathbb{R}, \rightarrow \theta x+(1-\theta) y\in C
$$

这个定义很容易推广到多个元素的情况,即
$$
\forall x_1,x_2,...,x_n\in C,\theta_1,\theta_2,..,\theta_n\in\mathbb{R},\rightarrow \theta_1 x_1+\theta_2 x_2+...+\theta_n x_n\in C
$$

其中$\theta_1+...+\theta_n=1$。$\theta_1 x_1+\theta_2 x_2+...+\theta_n x_n$被称为仿射组合

\subsection{子空间}
对于一个仿射集$ C$以及其内部任意一点$x_0$,定义集合$\mathrm{V}$
$$
V= C-x_0=\{x-x_0|x\in C\}
$$

这里$\mathrm{V}$是一个子空间。(满足性质:零向量属于子空间,子空间任意两个向量的加和,单向量与标量的乘积也在子空间里),仿射空间是子空间的一个\key{平移}。

对于矩阵$\mathrm{A}\in\mathbb{R}^{m\times n},b\in\mathbb{R}^n$,其解集$C=\{x|\mathrm{A}x=b\}$是一个仿射集。令集合$V=\{v|\mathrm{A}v=0\}$是一个子空间,并且有$\mathrm{A}x_0=b$,那么就有$C=V+x_0$。

\paragraph{举例}
考虑方程组$x_1+2x_2=3$,其含有解$x_1=-1,x_2=2$以及$x_1=1,x_2=1$,那么这两个解的线性组合也是原方程组的解。比如$x_1=0.8*-1+0.2*1=-0.6,x_2=0.8*2+0.2*1=1.8$也是方程的解。其齐次形式$x_1+2x_2=0$的任意解,比如$x_1=-2,x_2=1$加上一个特解$x_1=1,x_2=1$,也是原方程的解。

\subsection{仿射包affine hull}
一个集合C内所有点的仿射组合,被称为仿射包
$$
\mathrm{aff}C=\{\theta_1x_1+...+\theta_kx_k|x_1,...,x_k\in C,\theta_1+...+\theta_k=1\}
$$

仿射包$\mathrm{aff}C$是包含集合$C$中所有元素的最小的仿射集。

\subsection{凸集convex set}
对于集合C,如果有:
$$
\forall x,y\in C,\theta\in\mathbb{R},\theta>0 \rightarrow \theta x+(1-\theta) y\in C
$$

对比仿射集的定义,凸集要求系数$\theta>0$。同样可以推广到任意个点的情况。同样,仿射组合也对应``凸组合"

\subsection{凸包convex hull}
类似的定义,集合C的凸包,记为$\mathrm{conv}C$,是包含集合C中所有元素的凸组合的集合,是包含集合C的所有元素的最小的凸集:
$$
\mathrm{conv}C=\{\theta_1x_1+...+\theta_kx_k|x_i\in C,\theta_i\ge0,i=1...k,\theta_1+...+\theta_k=1\}
$$

\subsection{锥cone}
一个集合C被称为锥,如果对于$\forall x\in C,\theta\ge 0\rightarrow \theta x\in C$

一个集合C被称为\key{凸锥},如果对于$\forall x_1,x_2\in C,\theta_1,\theta_2\ge 0,\rightarrow \theta_1 x_1+\theta_2 x_2\in C$

锥组合:$\theta_1 x_1+...+\theta_k x_k,\theta_1,...,\theta_k\ge 0$。

锥包(conic hull): $\{\theta_1 x_1+...+\theta_kx_k|x_i\in C,\theta_i\ge0\}$

下图分别展示了一组离散点的锥包和一个非凸形状的锥包。

\begin{figure}[H]%
    \centering
    \begin{overpic}[width=0.6\linewidth]{fig/5.1.png}
    \end{overpic}
    \vspace{-3.5mm}
    \vspace{2mm}
\end{figure}

值得注意的是两个图中都绘制了一个原点。

\subsection{一些例子和应用}
\paragraph{超平面}一个超平面是满足$\{\mathrm{x}|a^\top \mathrm{x}=b\}$的集合,是一个凸集。
\paragraph{半空间}半空间是满足$\{\mathrm{x}|a^\top \mathrm{x}\le b\}$的集合,也是一个凸集。
\paragraph{多面体Polyhedra}一个多面体被定义为一组有限的线性等式与不等式的解。当然是凸的。
$$
\mathcal{P}=\{\mathrm{x}|a_j^\top \mathrm{x}\le b_j,j=1,...,m,c_k^\top \mathrm{x}=d_k,k=1,...,p\}
$$

\paragraph{球Ball}$B(\mathrm{x}_c,r)=\{\mathrm{x}|\lVert \mathrm{x}-\mathrm{x}_c \lVert_2\le r\}$ 这里半径大于0,范数也可以不是2。但不可以小于1,小于1无法满足范数的定义所满足的性质(三角不等式约束)。

\paragraph{范式球与范式锥}\footnote{https://blog.csdn.net/robert\_chen1988/article/details/80479813}

球心和半径都确定的范式球的定义为$\{\mathrm{x}|\lVert \mathrm{x}-\mathrm{x}_c \lVert\le r\}$

同样,定义范式锥:$C=\{(\mathrm{x},t)|\lVert \mathrm{x} \lVert \le t\}$

值得注意的是,范式锥比范式球高一维——范式锥的$r$也是维度之一。

如果在矩阵空间中定义,则有一些其他的凸结构例子:
\paragraph{半正定锥} \footnote{https://www.bilibili.com/read/cv4665074}
对称矩阵$\mathrm{S}^n$,半正定对称矩阵$\mathrm{S}_+^n$,正定对称矩阵$\mathrm{S}_{++}^n$都是凸的。(详见参考链接)

\subsection{正常锥proper cone以及广义不等性}
\paragraph{定义} 正常锥可以用于对锥中的元素定义一个偏序的关系。一个正常锥$K$是满足如下几个条件的锥:
\begin{enumerate}
    \item 凸的
    \item 闭的,即包含边界
    \item 非空的(solid),即具有非空内部。
    \item 定向的(pointed),含义是,锥的内部不能含有直线,例如:$x\in K and -x\in K\Rightarrow x=0$也就是说除非$x$就是原点,否则$x$与$-x$不能同时在集合内部。
\end{enumerate}

根据这个定义,就可以定义一个集合内部的偏序关系。在$\mathbb{R}^n$中的偏序拥有在$\mathbb{R}$中的有序的许多性质。
$$
\begin{aligned}
    x\preceq_K y \Leftrightarrow y-x\in K\\
    x\prec_K y\Leftrightarrow y-x\in \mathrm{int}K
\end{aligned}
$$

在$K=\mathbb{R}_+$的时候(也就是正的数轴),$\preceq_K$和$\le$是相同含义;

\paragraph{举例}
\begin{enumerate}
    \item n维的非负象限(orthant)$K=\mathbb{R}_+^n$是一个正常锥。此时$\mathrm{x}\preceq_K \mathrm{y}$的含义就是$\mathrm{x}$的每一个分量都小于$\mathrm{y}$,即$x_i<y_i,i=1...n$。
    \item 在由所有的$n$维对称矩阵构成的空间$\mathrm{S}^n$中,半正定锥$\mathrm{S}_+^n$是一个正常锥。此时$\mathrm{X}\preceq_K\mathrm{Y}$的含义是$\mathrm{X}-\mathrm{Y}$是半正定的。由于经常使用,对于半正定矩阵,可以忽略广义不等$\preceq_K$的下标。$\mathrm{X}\preceq\mathrm{Y}$就可以直接表示一种偏序关系。
    \item 在$[0,1]$区间上的非负多项式,也就是
    $$
    K=\{\mathrm{c}\in\mathbb{R}^n|c_1+c_2t+c_3t^2+...+c_nt^{n-1},t\in[0,1]\}
    $$此时对于两个向量$\mathrm{c},\mathrm{d},\mathrm{c}\preceq\mathrm{d}$当且仅当$c_1+c_2t+...+c_nt^{n-1}\le d_1+d_2t+...+d_nt^{n-1}$
\end{enumerate}

\subsection{极小元和最小元}
广义不等$\preceq$是不等$\le$的类比。有些属性是可以类比的,但有些属性不能类比。最常见的就是,在$\mathbb{R}$中,任意两个元素都是\key{可比}的。因此,在广义不等中,极小和最小的定义就变得复杂。

\paragraph{最小元minimum} 称$\mathrm{x}$是$S$中的最小元,如果$\forall \mathrm{y}\in S\Rightarrow \mathrm{x}\preceq_K \mathrm{y}$,显然,最小元是唯一的。使用\textbf{集合符号}来描述的话,就是:$\mathrm{x}\in S$是最小元当且仅当$S\subseteq \mathrm{x}+K$。

\paragraph{极小元minimal} 称$\mathrm{x}$是$S$中的极小元,如果$\mathrm{y}\preceq\mathrm{y} \Rightarrow \mathrm{y}=\mathrm{x}$。也就是说,如果$\mathrm{x},\mathrm{y}$可比,那么一定有$\mathrm{x}\preceq\mathrm{y}$。用集合符号描述就是:$\mathrm{x}\in S$是极小元当且仅当$(\mathrm{x}-K)\cap S=\{\mathrm{x}\}$。

\begin{figure}[H]%
    \centering
    \begin{overpic}[width=0.75\linewidth]{fig/5.2.png}
    \end{overpic}
    \vspace{-3.5mm}
    \vspace{2mm}
\end{figure}

\section{凸函数 convex funtion}
\subsection{定义}
函数是一个$\mathbb{R}^n \rightarrow \mathbb{R}$的一个映射。如果这个映射是凸的,当且仅当
$$
f(\theta x+(1-\theta)y)\le \theta f(x)+(1-\theta)f(y)
$$

也就是凸组合的函数值小于等于函数值的凸组合。严格凸就是当$x\neq y$的时候,$\le \Rightarrow <$

\begin{figure}[H]%
    \centering
    \begin{overpic}[width=0.5\linewidth]{fig/5.3.png}
    \end{overpic}
    \vspace{-3.5mm}
    \vspace{2mm}
\end{figure}

\paragraph{梯度(一阶导)定义}充要条件:$f(y)\ge f(x)+\nabla f(x)^\top (y-x)$,表示沿着任意一点做切线,函数图像都在切线的上方。

\paragraph{二阶导定义} 充要条件:$\nabla^2 f(x)\succeq 0$,也就是Hessian矩阵是半正定的。但值得注意的是,Hessian正定可以推出严格凸,但反之,严格凸无法推导出Hessian正定。比如$f(x)=x^4$,其Hessian在$x=0$处等于0。(但也满足半正定的要求)。

和前述最优性条件的区分:最优性条件是在特定的点处的性质,且满足二阶最优性条件的前提是满足一阶最优性条件;而凸性的一阶二阶条件是在整个函数上的性质。

\subsection{常见的凸函数}
\begin{enumerate}
    \item 指数函数$e^{ax}$,在$\mathbb{R}$上。
    \item 幂函数$x^a$,在$\mathbb{R}_{++}$上,也就是要求$x$是正数;当$a\ge 1\  or\ a\le0$的时候是凸的,在$[0,1]$上是凹的。
    \item 绝对值的幂函数$\lvert x \lvert^p,p\ge 1$在$\mathbb{R}$上是凸的
    \item 对数函数在$\mathbb{R}_{++}$上是凸的
    \item 熵函数也是凸的$x\log(x)$,但要注意定义域。(约定$0\log(0)=0$)
    \item 范数是凸的。
    \item 最大值函数$f(\mathrm{x})=\max(x_1,x_2,...,x_n)$是凸函数。这个函数无法求导,但是可以从定义角度来证明。
    \item 指数求和函数$f(\mathrm{x})=\log(e^{x_1}+e^{x_2}+...+e^{x_n})$是凸的。这个函数满足
    $$
    \max\{x_1,...,x_n\}\le f(\mathrm{x})\le \max\{x_1,...,x_n\}+\log(n)
    $$
    \item 几何平均$f(\mathrm{x})=(\prod_{i=1}^n)^{1/n}$是\key{凹}的。
    \item 行列式的对数函数$f(\mathrm{X})=\log\det(\mathrm{x})$是凸的。
\end{enumerate}

\subsection{常见的凸函数算子}
以下介绍一些常见的可以使得函数保持凸性的函数算子。
\begin{enumerate}
    \item 非负的加权和 $f=w_1f_1+...+w_mf_m,w_1,...,w_m\ge 0$
    \item 推广到无限项相加的情况,即为积分:如果对于$\forall y\in\mathcal{A}$,都有$f(x,y)$是凸的,且$w(y)\ge 0$,那么函数
    $$
    g(x)=\int_{\mathcal{A}}w(y)f(x,y)\mathrm{d}y
    $$
    关于$x$也是凸的。
    \item 和仿射变换组合:$f:\mathbb{R}^n\mapsto \mathbb{R},A\in\mathbb{R}^{n\times m},b\in \mathbb{R}^n,g:\mathbb{R}^m\mapsto\mathbb{R}$,那么有
    $$
    g(x)=f(\mathrm{A}x+b)
    $$
    如果$f$是凸的,那么$g$也是凸的。
    \item 分片最大值:$f(x)=\max\{f_1(x),f_2(x)\}$,如果$f_1,f_2$都是凸的,那么$f$也是凸的。
    \item 分片上确界:如果对于$\forall y\in\mathcal{A}$,都有$f(x,y)$是凸的,那么有
    $$
    g(x)=\sup_{y\in\mathcal{A}}f(x,y)
    $$
    也是凸的。(当然要满足定义域存在、上确界存在等等条件。)
    \item Quasi-convex\key{拟凸函数} $S_\alpha=\{x\in\mathrm{dom}f|f(x)\le\alpha\}$也是凸的(水平集)。拟凸函数:如果一个函数本身不是凸的,但是其水平集都是凸的,则称其为拟凸函数。\footnote{https://zhuanlan.zhihu.com/p/131604034}
    \item 对数凹函数:如果函数$f$满足$f(x)>0,\forall x\in\mathrm{dom}f$,且$\log f$是凹的,那么这就是一个对数凹函数。也就是说对于函数本身不要求是凹的,但是其对数是凹的,那么这种类型就是对数凹函数。
\end{enumerate}

\end{document}