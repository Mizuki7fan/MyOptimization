 \documentclass[a4paper]{D:/repositories/MyDGP/latex/PaperReadingLog}
\usepackage{caption}
\usepackage{overpic}
\usepackage{float}

\usepackage{enumitem} 
\usepackage{amsfonts,amssymb,amsmath}
\usepackage[linesnumbered,ruled,lined]{algorithm2e}%%伪代码
\usepackage{multicol}
\usepackage{wrapfig}
\setmainfont{Arial}

\begin{document}

\PaperInfo
{基于线搜索的最优化算法}
{朱}
{天宇}
{}
最优化方法从分类级别从高到低的顺序来说,可以分为几层:

基于迭代下降的方法-基于线搜索的方法-基于梯度的方法-具体的各种方法

本节内容主要介绍基于线搜索的方法。

\section{Prototype}
需要准备\key{\textbf{初始点}},\key{\textbf{搜索方向}},\key{\textbf{步长}}
\begin{figure}[H]%使用H表示Here,不对图片排版!
    \centering
    \begin{overpic}[width=0.99\linewidth]{fig/1.1.png}
    \end{overpic}
    \vspace{-3.5mm}
    \vspace{2mm}
\end{figure}

\section{基于梯度的方法}
基于梯度的方法的重点是\textbf{如何构造搜索方向}。其使用\textbf{负梯度方向}来构造搜索方向$\mathrm{d}^{(k)}$。主要步骤有3步:
\begin{enumerate}
    \item 初始化,选择一个合适的起点$\mathrm{x}^{(0)}\in\mathbb{R}^n$,$k\leftarrow 0$
    \item 搜索方向$\mathrm{d}^{(k)}\leftarrow -\key{\mathrm{H}_k}\nabla f(\mathrm{x}^{(k)})$,其中$\mathrm{H}_k$是一个正定、对称矩阵
    \item 确定步长因子,这通过解一个一维的最优化问题$\min_{\alpha\ge 0}f(\mathrm{x}^{(k)}+\alpha \mathrm{d}^{(k)})$来获得。
    \item 更新:$k\leftarrow k+1, \mathrm{x}^{(k+1)}\leftarrow \mathrm{x}^{(k)}+\alpha_k\mathrm{d}^{(k)}$
\end{enumerate}

\subsection{$\mathrm{H}_k$是单位阵,最速下降法}
此时$\mathrm{d^{(k)}}=-\nabla f(\mathrm{x}^{(k)})$。对于无约束的最优化问题$\min_{\mathrm{x}\in \mathrm{R}^n}$,泰勒展开有:
$$
f(\mathrm{x})=f(\mathrm{x}^{(k)})+\nabla f(\mathrm{x}^{(k)})^\top(\mathrm{x}-\mathrm{x}^{(k)})+O(\lVert \mathrm{x}-\mathrm{x}^{(k)} \lVert ^2)
$$
当$\mathrm{d^{(k)}}=-\nabla f(\mathrm{x}^{(k)})$的时候,总是存在一个足够小的$\alpha_k$,使得
$$
f(\mathrm{x}^{(k)}+\alpha_k\mathrm{d}^{(k)})<f(\mathrm{x}^{(k)})
$$
这通过将$f(\mathrm{x}^{(k)}+\alpha_k\mathrm{d}^{(k)})$进行泰勒展开后就可以验证。

虽然$\mathrm{H}_k$为单位阵的时候,搜索方向是函数下降最快的方向,但这并不意味着这么做结果能达到最好。


\subsection{$\mathrm{H}_k=(\nabla^2f(\mathrm{x}^{(k)}))^{-1}$是Hessian矩阵的逆矩阵}
这里\textbf{要求Hessian矩阵是正定的},也就是每个特征值都大于0。同样,将$f(\mathrm{x}^{(k)}+\alpha_k\mathrm{d}^{(k)})$在$\mathrm{x}^k$处进行二阶的泰勒展开:
$$
\begin{aligned}
    f(\mathrm{x}^{(k)}+\alpha_k\mathrm{d}^{(k)})&=f(\mathrm{x}^{(k)})+\nabla f(\mathrm{x}^{(k)})^\top(\mathrm{x}^{(k)}+\alpha_k\mathrm{d}^{(k)}-\mathrm{x}^{(k)})\\
    &+\frac{1}{2}(\mathrm{x}^{(k)}+\alpha_k\mathrm{d}^{(k)}-\mathrm{x}^{(k)})^\top\nabla^2 f(\mathrm{x}^{(k)})(\mathrm{x}^{(k)}+\alpha_k\mathrm{d}^{(k)}-\mathrm{x}^{(k)})\\
    &+O(\lVert (\mathrm{x}^{(k)}+\alpha_k\mathrm{d}^{(k)}-\mathrm{x}^{(k)}) \lVert ^3)\\
    &= f(\mathrm{x}^{(k)})+\nabla f(\mathrm{x}^{(k)})^\top(\alpha_k\mathrm{d}^{(k)})\\
    &+\frac{1}{2}(\alpha_k\mathrm{d}^{(k)})^\top\nabla^2 f(\mathrm{x}^{(k)})(\alpha_k\mathrm{d}^{(k)})\\
    &+O(\lVert (\alpha_k\mathrm{d}^{(k)}) \lVert ^3)\\
    &= f(\mathrm{x}^{(k)})+\alpha_k\nabla f(\mathrm{x}^{(k)})^\top\key{\mathrm{d}^{(k)}}\\
    &+\frac{1}{2}\alpha_k^2\key{\mathrm{d}^{(k)\top}}\nabla^2 f(\mathrm{x}^{(k)})\key{\mathrm{d}^{(k)}}\\
    &+O(\lVert (\alpha_k\mathrm{d}^{(k)}) \lVert ^3)\\
\end{aligned}
$$
在这个式子中,一阶展开为
$$
\begin{aligned}
    \alpha_k\nabla f(\mathrm{x}^{(k)})^\top\key{\mathrm{d}^{(k)}}&=\key{-}\alpha_k\nabla f(\mathrm{x}^{(k)})^\top\key{(\nabla^2f(\mathrm{x}^{(k)}))^{-1}}\nabla f(\mathrm{x}^{(k)})\\
    &=\key{-}\alpha_k\nabla f(\mathrm{x}^{(k)})^\top G_k^{-1}\nabla f(\mathrm{x}^{(k)})
\end{aligned}
$$
这是一个\textbf{二次型},中间的Hessian矩阵的逆矩阵也是正定的。所以此项的值为负。

对于二阶展开
$$
\begin{aligned}
    \frac{1}{2}\alpha_k^2\key{\mathrm{d}^{(k)\top}}\nabla^2 f(\mathrm{x}^{(k)})\key{\mathrm{d}^{(k)}}&=\frac{1}{2}\alpha_k^2\nabla f(\mathrm{x}^{(k)})^\top (G_k^{-1})^\top G_k G_k^{-1} \nabla f(\mathrm{x}^{(k)})\\
    &=\frac{1}{2}\alpha_k^2\nabla f(\mathrm{x}^{(k)})^\top (G_k^{-1})^\top \nabla f(\mathrm{x}^{(k)})
\end{aligned}
$$
这也是一个二次型,且由于Hessian矩阵是正定对称的,所以$(G_k^{-1})^\top=G_k^{-1}$,所以第二项的中间的矩阵和第一项的中间的矩阵是一样的。而由于$\alpha_k$是一个很小的正数,所以$\frac{1}{2}\alpha_k^2-\alpha_k<0$,所以一阶项与二阶项的和是小于0的。

因此同样有
$$
f(\mathrm{x}^{(k)}+\alpha_k\mathrm{d}^{(k)})<f(\mathrm{x}^{(k)})
$$

实际上,$d^k$是函数在$x^k$点处二次展开的极小点。

\subsection{确定步长因子$\alpha$}
一般来说,$\alpha$肯定是要尽可能的大,这样下降就会比较快。但如果$\alpha$太大,可能根本无法保证下降。因此要通过优化以下这个一维问题来确定最大步长。
$$
\min_{\alpha\ge 0}\varphi(\alpha)=f(\mathrm{x}^{(k)}+\alpha\mathrm{d}^{(k)})
$$

如果以这个方程的最优解作为步长,此时就称为"\textbf{精确的一维搜索}"。通常用黄金分割法或者插值迭代法(指先采样若干个点,插值出一个函数再求最优解)来处理。

与之相区别,"\textbf{非精确的一维搜索}"只希望找到某些条件的粗略近似解,可以提高整体的计算效率。

设$\bar{\alpha}_k$是使得$f(\mathrm{x}^{(k)}+\alpha\mathrm{d}^{(k)})=f(\mathrm{x}^{(k)})$的最小的正数。由于预设函数$\varphi$满足\key{单峰性},所以就需要从区间$[0,\bar{\alpha}_k]$中寻找一个可以接受的步长$\alpha$作为步长。比如通过Glodstein条件就可以快速找到一个令人满意的值。

\paragraph{Goldstein条件}
$$
\begin{aligned}
    \varphi(\alpha)&\le \varphi(0)+\rho\alpha\varphi'(0)\\
    \varphi(\alpha)&\ge \varphi(0)+(1-\rho)\alpha\varphi'(0)
\end{aligned}
$$
其中$\rho\in(0,\frac{1}{2})$,是一个固定的参数。

上式可以保证$\alpha$满足一定的下降率。因为$\varphi(0)$表示$f$在$\mathrm{x}^{(k)}$的取值,$\varphi(0)+\rho\alpha\varphi'(0)$表示关于$\alpha$的一条直线,而$\varphi(\alpha)$要在这条直线的下方($\varphi'(0)=\nabla f^\top\mathrm{d}^{(k)}<0$,斜率小于0)。而$\varphi(\alpha)$要更小。

同理,下面的式子则表示图中蓝色的线。由于$\rho\in(0,\frac{1}{2})$,所以$1-\rho>\rho$,所以蓝色的线更加陡一点。添加一条蓝色的线是为了防止$\alpha$无线接近于0,防止下降太慢。

\begin{figure}[H]%使用H表示Here,不对图片排版!
    \centering
    \begin{overpic}[width=0.5\linewidth]{fig/1.2.png}
    \end{overpic}
    \vspace{-3.5mm}
    \vspace{2mm}
\end{figure}
这样,$\alpha$就落在图中的红圈内部。

显然,从图中就可以看出Glodstein条件的问题。在图中,最好的$\alpha$的取值被忽略了。最好能找到一个不排除最优解的区域。

\paragraph{Wolfe-Powell条件}
$$
\begin{aligned}
    \varphi(\alpha)&\le \varphi(0)+\rho\alpha\varphi'(0)\\
    \varphi'(\alpha)&\ge \sigma\varphi'(0)
\end{aligned}
$$
其中,$\sigma\in(\rho,1)$,是另一个固定的参数。上面的约束和Goldstein条件相同。下面的约束对蓝色直线的斜率进行了约束。具体可以参见下图。因为在最优解的地方,切线的斜率为0;$\sigma\varphi'(0)$是一个小于0的值,因此约束$\varphi'(\alpha)\ge \sigma\varphi'(0)$是可以保证$\alpha$最小取值是在最优解左侧的。但保证一个负的最小值又可以防止$\alpha$太靠近0。

\begin{figure}[H]
    \centering
    \begin{overpic}[width=0.5\linewidth]{fig/1.3.png}
    \end{overpic}
    \vspace{-3.5mm}
    \vspace{2mm}
\end{figure}
有时候,也可以使用这个条件:
$$
\lvert \varphi'(\alpha) \lvert \le -\sigma\varphi'(0)
$$

\paragraph{基于Wolfe-Powell准则的一维搜索算法}
如果从区间内随便选点的话很容易代价过大,因此需要根据准则,定制一套明确的算法。
\begin{enumerate}
    \item \key{初始化}\\规定初始搜索区间为$[0,\bar{\alpha}]$;\\选择固定的$\rho\in(0,1/2);\sigma\in(\rho,1)$,\\令$\varphi_0=\varphi(0)=f(\mathrm{x^{(k)}});\varphi_0'=\varphi'(0)=\nabla f(\mathrm{x}^{(k)})^\top\mathrm{d}^{(k)}$,\\定义$a_1=0,a_2=\bar{\alpha}$作为$\alpha$取值的上下界,$\varphi_1=\varphi_0,\varphi_1'=\varphi_0'$,\\选择适当的$\alpha\in(a_1,a_2)$
    \item \key{调整下界}\\
    计算$\varphi(\alpha)=f(\mathrm{x}^{(k)}+\alpha\mathrm{d}^{(k)})$,如果满足$\varphi(\alpha)\le \varphi(0)+\rho\alpha\varphi'(0)$,则进入下一步;\\
    否则更新$\hat{\alpha}$
    $$
    \hat{\alpha}=a_1+\frac{1}{2}\frac{(a_1-\alpha)^2\varphi_1'}{(\varphi_1-\varphi)-(a_1-\alpha)\varphi_1'}
    $$
    这实际上是对$\varphi_1,\varphi_1',\varphi$三点构造的二次插值多项式的极小点。\\
    令$a_2=\alpha,\alpha=\hat{\alpha}$,缩小取值范围,并重复此步。
    \item \key{调整上界}\\
    计算$\varphi'=\varphi'(\alpha)=\nabla f(\mathrm{x}^{(k)}+\alpha\mathrm{d}^{(k)})^\top\mathrm{d}^{(k)}$,如果满足$\varphi'(\alpha)\ge\sigma\varphi'(0)$,则输出$\alpha_k=\alpha$,算法结束。\\
    否则更新$\hat{\alpha}$
    $$
    \hat{\alpha}=\alpha-\frac{(a_1-\alpha)\varphi'}{\varphi_1'-\varphi'}
    $$
    这同样是二次插值的极小点。\\
    令$a_1=\alpha,\alpha=\hat{\alpha},\varphi_1=\varphi,\varphi_1'=\varphi'$,返回上一步重新调整下界。
\end{enumerate}

\subsection{全局收敛与局部收敛}
指初始点选择的范围大小。以上的方法都满足全局收敛的要求。
\paragraph{全局收敛}从任意初始点出发,都可以收敛到最优解,称为全局收敛。
\paragraph{局部收敛}初始点必须在最优解的附近。

\section{最速下降法}
最速下降法以负梯度方向作为下降方向。其满足全局收敛性,也就是说任意一个点采用最速下降法,其梯度方向都会趋向于0向量。具体迭代过程可以参考这张图:
\begin{figure}[H]
    \centering
    \begin{overpic}[width=0.5\linewidth]{fig/1.4.png}
    \end{overpic}
    \vspace{-3.5mm}
    \vspace{2mm}
\end{figure}
图中,每一次移动的方向都是当前点$\mathrm{x}_n$的负梯度方向。而每次移动的长度则由线搜索来决定。

在图中,$\mathrm{g}^{k+1}$和$\mathrm{g}^{k}$是相互垂直的。这是因为图中的线搜索是``精确的一维搜索'',那么这两个梯度就是互相垂直的。呈现如图的情况。如果不是精确的一维搜索,则有可能更快也可能变慢。

这种方法,点的移动轨迹是一个``Z"字型。如果采用一些策略,比如改变方向或者改变步长,则可能可以更快的收敛到结果。因此,最速下降法的收敛取决于函数等值线的形状。

最速下降法通常只有线性的收敛速度,对于无约束的问题来说,这是远远不够的。比如对于著名的测试函数Rosenbrock function,这个函数的等值线是一个长短轴比例非常悬殊的椭圆形。
$$
f(x_1,x_2)=100(x_2-x_1^2)^2+(1-x_1)^2
$$
这个函数的图像是:
\begin{figure}[H]
    \centering
    \begin{overpic}[width=0.5\linewidth]{fig/1.5.png}
    \end{overpic}
    \vspace{-3.5mm}
    \vspace{2mm}
\end{figure}

\section{牛顿法}
牛顿法的下降方向是Hessian矩阵的逆矩阵乘以梯度方向。设$f(\mathrm{x})$是二次可微的实函数,在$\mathrm{x}^{(k)}$处做二阶的泰勒展开,有
$$
f(\mathrm{x}^{(k)}+\mathrm{s})\approx q^{(k)}(\mathrm{s})=f(\mathrm{x}^{(k)})+\mathrm{g}^{(k)^\top}\mathrm{s}+\frac{1}{2}\mathrm{s}^\top G_k\mathrm{s}
$$
其中$\mathrm{g}^{(k)}=\nabla f(\mathrm{x}^{(k)})$,是梯度;$G_k=\nabla^2f(\mathrm{x}^{(k)})$,是Hessian矩阵;将$q^{(k)}(\mathrm{s})$极小化,就可以得到
$$
\mathrm{s}=-G_k^{-1}\mathrm{g}^{(k)}
$$
将$\mathrm{s}$作为搜索方向,就称为\textbf{牛顿方向}。

对于正定二次函数
$$
f(\mathrm{x})=\frac{1}{2}\mathrm{x}^\top G\mathrm{x}-\mathrm{c}^\top\mathrm{x}
$$
对于任意的$\mathrm{x}$,都有$\nabla^2f(\mathrm{x})=G$。那么
$$
\mathrm{d}^{(0)}=-G^{-1}\nabla f(\mathrm{x}^{(0)})=-G^{-1}(G\mathrm{x}^{(0)}-\mathrm{c})=-(\mathrm{x}^{(0)}-\mathrm{x}^\star)
$$
使用$\mathrm{x}^\star=G^{-1}\mathrm{c}$表示问题的最优解。如果$\mathrm{x}^{(0)}\neq\mathrm{x}^\star$,那么取步长$\alpha_0=1$,就有$\mathrm{x}^{(1)}=\mathrm{x}^{(0)}+\alpha_0\mathrm{d}^{(0)}=\mathrm{x}^\star$,也就是说经过一次迭代之后就可以达到最优解,而最速下降法需要多次迭代。最速下降法下降的速度取决于Hessian矩阵的特征值的差异,也对应椭圆的长短轴。

基于以上事实,可以认为对于一般的非线性函数,在迭代中取这样的搜索方向也是比较合适的。

特别的,令步长$\alpha\equiv 1$,迭代公式就变为
$$
\mathrm{x}^{(k+1)}=\mathrm{x}^{(k)}-G_k^{-1}\mathrm{g}^{(k)}
$$
这就是经典的牛顿法更新公式。

\paragraph{性质}
\begin{enumerate}
    \item 对于正定二次函数,牛顿法可以一步达到最优解。对于非二次函数,牛顿法无法保证有限次迭代;但是在初始点靠近极小点的时候,收敛速度一般会很快,因为此时目标函数可以较好的用二次函数来模拟。
    \item 牛顿法是\key{局部收敛}的,在满足一定条件的情况下,收敛速率是二阶的。
\end{enumerate}
收敛性证明(略,详见视频)

\subsection{改进策略}
\paragraph{阻尼牛顿法}
阻尼牛顿法对步长的步骤进行改进,采用一维搜索来确定步长,从而避免局部收敛。
\begin{enumerate}
    \item \key{初始化}初始点$\mathrm{x}^{(0)}$,设置终止误差$\epsilon>0$
    \item 计算$\mathrm{g}^{(k)}=\nabla f(\mathrm{x}^{(k)})$,如果$\lVert \mathrm{g}^{(k)} \lVert <\epsilon$,则算法结束
    \item \key{解线性方程组}$G_k\mathrm{d}=-\mathrm{g}^{(k)}$,获得牛顿方向$\mathrm{d}^{(k)}$
    \item 采用一维搜索确定$\alpha_k$,更新$\mathrm{x}$,并且回到第1步
\end{enumerate}
这种方法可以缓解经典牛顿法局部收敛的缺点,但是需要解线性方程组,且无法确保$\mathrm{d}^{(k)}$满足全局收敛的条件。

一般来说早期会使用一维搜索确定步长,后期则固定$\alpha_k=1$

\paragraph{Goldstein \& Price}
$$
\mathrm{d}^{(k)}=\left\{
\begin{aligned}
    &-G_k^{-1}\mathrm{g}^{(k)},\quad if\ \cos\theta_k>\eta\\
    &-\mathrm{g}^{(k)},\quad otherwise
\end{aligned}
\right.
$$
也就是根据$\theta$的取值来决定使用最速下降还是牛顿法。可以保证全局收敛性。缺点是不太好分析收敛效率。

\paragraph{Levenberg,Marquardt,Goldfeld,et.al}
$$
(G_k+\mu_kI)\mathrm{d}^{(k)}=-\mathrm{g}^{(k)}
$$
L-M方法的优势是,在阻尼牛顿法中,需要$G_k\mathrm{d}=-\mathrm{g}^{(k)}$,这么做可以便于分析。是一种牛顿方向和负梯度方向的综合,可以通过$\mu_k$调节倾向于牛顿方向还是负梯度方向。$\mu_k$取值可以大于$G_k$的最小特征值,使其变成正定矩阵。

\paragraph{负曲率方向法}
如果方向$\mathrm{d}$,满足
$$
\mathrm{d}^\top\nabla^2f(\mathrm{x})\mathrm{d}<0
$$
则称其为负曲率方向。

当Hessian矩阵不正定的时候,可以用负曲率方向来修正牛顿法。

\paragraph{拟牛顿法}
运用牛顿法需要计算二阶导数,代价高,且Hessian矩阵不一定正定,甚至奇异。如果使用不含二阶导的矩阵$H_k$来近似牛顿法中的Hessian矩阵的逆,这种方法被称为\key{拟牛顿法}。

构造近似矩阵有不同的方法,因此拟牛顿法也有多种类型。
\end{document}