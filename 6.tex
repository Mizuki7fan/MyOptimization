\documentclass[a4paper]{D:/MyRepo/Script/latex/PaperReadingLog}
\usepackage{caption}
\usepackage{overpic}
\usepackage{float}

\usepackage{enumitem} 
\usepackage{amsfonts,amssymb,amsmath}

\usepackage{algorithm}%%算法伪代码,支持中文
\usepackage{algorithmic}
\setmainfont{Arial}

\begin{document}

\PaperInfo
{6.凸优化}
{朱}
{天宇}
{}

\section{问题建模}
凸优化问题一般要解决
\begin{equation}
    \label{equ:1}
    \begin{aligned}
        \min\quad&f_0(\mathrm{x})\\
        s.t.\quad&f_i(\mathrm{x})\le0,\quad i=1,...m\\
        &h_j(\mathrm{x})=0,\quad j=1,...p
    \end{aligned}
\end{equation}
这其中,\begin{enumerate}
    \item $\mathrm{x}\in\mathbb{R}^n$,是优化变量
    \item $f_0:\mathbb{R}^n\rightarrow \mathbb{R}$,被称为目标函数,或者是cost function
    \item $f_i(\mathrm{x})\le0$,不等式约束
    \item $f_i:\mathbb{R}^n\rightarrow \mathbb{R}$,被称为不等式约束函数
    \item $h_j(\mathrm{x})=0$,等式约束
    \item $h_j:\mathbb{R}^n\rightarrow \mathbb{R}$,被称为等式约束函数
\end{enumerate}

\paragraph{可行域、可行点}
可行域就是所有约束的可行域的交集;如果可行域$\mathcal{D}$非空,则称\ref{equ:1}为可行的(feasible)。


\end{document}