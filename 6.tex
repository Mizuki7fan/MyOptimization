\documentclass[a4paper]{D:/MyRepo/Script/latex/PaperReadingLog}
\usepackage{caption}
\usepackage{overpic}
\usepackage{float}

\usepackage{enumitem} 
\usepackage{amsfonts,amssymb,amsmath}

\usepackage{algorithm}%%算法伪代码,支持中文
\usepackage{algorithmic}
\setmainfont{Arial}

\begin{document}

\PaperInfo
{6.凸问题和KKT条件}
{朱}
{天宇}
{}

\section{凸问题建模和性质}
凸优化问题一般要解决
\begin{equation}
    \label{equ:1}
    \begin{aligned}
        \min\quad&f_0(\mathrm{x})\\
        s.t.\quad&f_i(\mathrm{x})\le0,\quad i=1,...m\\
        &a_j^\top \mathrm{x}=0,\quad j=1,...p
    \end{aligned}
\end{equation}

这其中,\begin{enumerate}
    \item $\mathrm{x}\in\mathbb{R}^n$,是优化变量
    \item $f_0:\mathbb{R}^n\rightarrow \mathbb{R}$,被称为目标函数,或者是cost function
    \item $f_i(\mathrm{x})\le0$,不等式约束
    \item $f_i:\mathbb{R}^n\rightarrow \mathbb{R}$,被称为不等式约束函数
    \item $a_j^\top \mathrm{x}=0$,线性等式约束
\end{enumerate}

\paragraph{凸问题的条件} 在\ref{equ:1}中,如果$f_0,f_1,...,f_m$都是凸函数,那么这就是一个凸优化问题。需要注意的是,等式约束\textbf{必须是}线性约束。

\paragraph{可行域、可行点}
可行域就是所有约束的可行域的交集;如果可行域$\mathcal{D}$非空,则称\ref{equ:1}为可行的(feasible)。在实际问题之前,需要先判定可行域是否非空。

\paragraph{最优值 $v^\star$ 最优解 $\mathrm{x}^\star$}  在可行域中,使得$f_0(\mathrm{x})$的下确界所对应的值。如果问题是不可行的,则$v^\star=\infty$。最优值是一个下确界,不能保证最优解存在。$f_0(\mathrm{x}^\star)=v^\star$。局部最优的定义和之前的也是相似的。凸优化问题的任意局部最优解也是全局最优解。

\subsection{一阶最优性条件}
如果目标$f_0$是可微的,那么点$\mathrm{x}$是最优解,\key{当且仅当}对于任意定义域的点$\mathrm{y}$,满足
$$\nabla f_0(\mathrm{x})^\top (\mathrm{y}-\mathrm{x})\ge 0$$。

对于无约束的情况,这个条件退化为
$$
\nabla f_0(\mathrm{x})=0
$$

对于\textbf{仅包括等式约束}的情况,问题可以写成:
$$
\begin{aligned}
    \min\quad&f_0(\mathrm{x})\\
    s.t.\quad&\mathrm{A}\mathrm{x}=b
\end{aligned}
$$

此时最优性条件可以写成
$$
\nabla f_0(\mathrm{x})^\top \mathrm{u}\ge 0,\forall \mathrm{u}\in\mathcal{N}(\mathrm{A})
$$

这里$\mathcal{N}(\mathrm{A})$表示矩阵的零空间,也就是使得$\mathrm{A}\mathrm{x}=0$的解空间。仅包含等式约束的凸优化问题的解与等式约束的零空间正交。

\key{相对于一般的优化问题,一阶最优性条件仅仅是充分条件,而对于凸优化,一阶最优性条件是充要条件。\footnote{https://www.zhihu.com/question/24641575/answer/507237591}}

\section{具体的凸优化问题}
\subsection{线性规划(LP)}
\begin{equation}
    \label{equ:2}
    \begin{aligned}
        \min\quad&\mathrm{q}^\top \mathrm{x}+\mathrm{r}\\
        s.t.\quad&\mathrm{G}\mathrm{x}\le \mathrm{h}\\
        &\mathrm{A}\mathrm{x}=\mathrm{b}
    \end{aligned}
\end{equation}
线性规划问题的目标函数和约束都是线性的。通常在目标函数的$r$可以忽略。其求解的基本思想给其他凸优化问题很多启发。

\subsection{二次规划(QP)}
\begin{equation}
    \label{equ:3}
    \begin{aligned}
        \min\quad&\frac{1}{2}\mathrm{x}^\top\mathrm{P}\mathrm{x}+\mathrm{q}^\top\mathrm{x}+\mathrm{r}\\
        s.t.\quad&\mathrm{G}\mathrm{x}\le \mathrm{h}\\
        &\mathrm{A}\mathrm{x}=\mathrm{b}
    \end{aligned}
\end{equation}

\subsection{二次约束的二次规划(QCQP)}
\begin{equation}
    \label{equ:3}
    \begin{aligned}
        \min\quad&\frac{1}{2}\mathrm{x}^\top\mathrm{P}_0\mathrm{x}+\mathrm{q}_0^\top\mathrm{x}+\mathrm{r}_0\\
        s.t.\quad&\frac{1}{2}\mathrm{x}^\top\mathrm{P}_i\mathrm{x}+\mathrm{q}_i^\top\mathrm{x}+\mathrm{r}_i\le0,i=1,...,m\\
        &\mathrm{A}\mathrm{x}=\mathrm{b}
    \end{aligned}
\end{equation}

\subsection{二阶锥规划(SOCP)}
\begin{equation}
    \label{equ:4}
    \begin{aligned}
        \min\quad&f^\top \mathrm{x}\\
        s.t.\quad&\lVert \mathrm{L}_i\mathrm{x}+\mathrm{g}_i\lVert_2\le \mathrm{c}_i^\top\mathrm{x}+\mathrm{d}_i,i=1,...,m\\
        &\mathrm{A}\mathrm{x}=\mathrm{b}
    \end{aligned}
\end{equation}

``二阶锥''体现在其约束上。二阶锥规划的目标函数是\textbf{线性}的。

在$\mathrm{c}_i=0$的时候,SOCP是QCQP的一个特殊情形。但实际上,SOCP有更好的模型化能力,也就是说更加``广义'',任意的QCQP都可以被归纳为一种SOCP的特殊情况。

\subsection{QCQP转为SOCP}
引入辅助变量$\mathrm{y}$,使得$\frac{1}{2}\mathrm{x}^\top\mathrm{P}_0\mathrm{x}+\mathrm{q}_0^\top\mathrm{x}+\mathrm{r}_0\le\mathrm{y}$,那么QCQP问题\ref{equ:3}就可以写成
\begin{equation}
    \label{equ:5}
    \begin{aligned}
        \min_{\mathrm{x},\mathrm{y}}\quad&\mathrm{y}\\
        % \min\quad&\frac{1}{2}\mathrm{x}^\top\mathrm{P}_0\mathrm{x}+\mathrm{q}_0^\top\mathrm{x}+\mathrm{r}_0\\
        s.t.\quad&\frac{1}{2}\mathrm{x}^\top\mathrm{P}_i\mathrm{x}+\mathrm{q}_i^\top\mathrm{x}+\mathrm{r}_i\le0,i=1,...,m\\
        &\frac{1}{2}\mathrm{x}^\top\mathrm{P}_0\mathrm{x}+\mathrm{q}_0^\top\mathrm{x}-\mathrm{y}+\mathrm{r}_0\le0\\
        &\mathrm{A}\mathrm{x}=\mathrm{b}
    \end{aligned}
\end{equation}

问题\ref{equ:5}仍然是一个QCQP问题。但是\ref{equ:5}的目标函数已经是一个线性约束了(SOCP的目标函数是线性函数),接下来需要将约束变成SOCP的形式就可以了。

对于一个二次约束$\frac{1}{2}\mathrm{x}^\top\mathrm{P}\mathrm{x}+\mathrm{q}^\top\mathrm{x}+\mathrm{r}\le0$,由于$\mathrm{P}\in S^n_+$,即是半正定的,故而可以找到一个根$\mathrm{L}$,使得$\mathrm{L}\mathrm{L}=\mathrm{P}$。令
$$
\widetilde{\mathrm{L}}=\left[
    \begin{array}[]{c}
        \mathrm{L} \\ \mathrm{q}^\top
    \end{array}
\right],\quad\widetilde{\mathrm{g}}=\left[
    \begin{array}[]{c}
        0 \\ \vdots \\ 0 \\ \mathrm{r}+\frac{1}{2}
    \end{array}
\right]\in\mathbb{R}^{n+1}
$$

这里,$\widetilde{\mathrm{L}}$是一个$(n+1)\times n$的矩阵,$\widetilde{\mathrm{g}}$是一个$n+1$维。

从而有
$$
\lVert \widetilde{\mathrm{L}}\mathrm{x}+\widetilde{\mathrm{g}}\lVert_2 \le -(\mathrm{q}^\top \mathrm{x}+\mathrm{r}-\frac{1}{2})
$$

从而转为了SOCP的形式。

\begin{figure}[H]%
    \centering
    \begin{overpic}[width=0.5\linewidth]{fig/6.1.png}
    \end{overpic}
    \vspace{-3.5mm}
    \vspace{2mm}
\end{figure}

\section{凸问题的特殊性}
\subsection{拉格朗日函数 Lagrangian}
回顾本章开头,凸优化问题的通式\ref{equ:1}
\begin{equation}
    \label{equ:6}
    \begin{aligned}
        \min\quad&f_0(\mathrm{x})\\
        s.t.\quad&f_i(\mathrm{x})\le0,\quad i=1,...m\\
        &h_j(\mathrm{x})=0,\quad j=1,...p
    \end{aligned}
\end{equation}

记录符号$\mathcal{D}=\cap^m_{i=0}\mathrm{dom}f_i \cap \cap_{j=1}^p\mathrm{dom}h_j$为问题的定义域。也就是约束的定义域的交集。

定义\key{拉格朗日函数}:
$$
L(\mathrm{x},\mathrm{\lambda},\mathrm{\nu})=f_0(\mathrm{x})+\sum_{i=1}^m\lambda_if_i(\mathrm{x})+\sum_{j=1}^p\mathrm{\nu}_jh_j(\mathrm{x})
$$

函数$L$的定义域为$\mathrm{dom}L=\mathcal{D}\times\mathbb{R}^m\times\mathbb{R}^p$。如果$\mathrm{x}$被固定住,那么$L$关于$\lambda,\nu$是线性的。$\lambda,\nu$被称为拉格朗日乘子,或者叫做\textbf{对偶变量}。

\subsection{拉格朗日对偶函数}
定义拉格朗日对偶函数$g:\mathbb{R}^m\times\mathbb{R}^p\rightarrow \mathbb{R}$:
\begin{equation}
    \label{equ:7}
    g(\lambda,\nu)=\inf_{\mathrm{x}\in\mathcal{D}} L(\mathrm{x},\lambda,\nu)=\inf_{\mathrm{x}\in\mathcal{D}}(f_0(\mathrm{x})+\sum_{i=1}^m\lambda_if_i(\mathrm{x})+\sum_{j=1}^p\mathrm{\nu}_jh_j(\mathrm{x}))    
\end{equation}

值得注意的是,对偶函数$g$是在$\mathrm{x}$的整个定义域$\mathcal{D}$中,对$L$取下确界获得的。$\mathcal{D}$是由约束函数的定义域的交集构成,并不要求一定要满足约束。也就是说,可能存在$\mathrm{x}_0\in\mathcal{D}$使得某个$f_i(\mathrm{x})>0$。在取完下确界$\inf$之后,$\mathrm{x}$就被消掉了。

由于拉格朗日对偶函数是关于$\lambda,\nu$线性的,故而其一定是凸的,无论原始问题\ref{equ:6}是不是凸的。

\paragraph{最优值的下界}记$v^\star$是原始问题\ref{equ:6}的最优值(不是最优解,是最优值),那么对于任意的$\lambda\ge0,\nu$,必然有
$$
g(\lambda,\nu)\le v^\star
$$

这样就为原问题的最优值确定了一个\textbf{下界}。尽管这是一个粗糙的下界,这个下界的取值可能是$-\infty$。为了获得一个\textbf{最好的下界},就需要解优化问题:
\begin{equation}
    \label{equ:8}
    \begin{aligned}
        \max\quad&g(\lambda,\nu)\\
        s.t.\quad&\lambda\ge 0
    \end{aligned}        
\end{equation}

这个问题被称为Lagrange dual problem,相应的,原问题\ref{equ:6}被称为原问题(primal problem)。

整个逻辑思路为:原问题\ref{equ:6}$\Rightarrow$拉格朗日函数\ref{equ:7}$\Rightarrow$对偶问题\ref{equ:8}

\paragraph{原问题和对偶问题}
对偶问题\ref{equ:8}总是一个凸问题,无论原问题是否是凸的;使用$\lambda^\star,\nu^\star$表示对偶问题的最优解,对于对偶问题的最大值$g^\star$,显然有
$$
g^\star\le v^\star
$$

这条性质被称为\key{弱对偶性}。对偶问题的最大值小于原问题的最小值。\textbf{对偶问题和原问题之间存在一个gap}。在如果gap等于0,则对偶问题的最优值就等于原问题的最优值。

\paragraph{Slater's condition}
满足$g^\star=v^\star$,这被称为强对偶性。强对偶性是一个比较高的条件,对于一个凸的原问题,还需要对其附加额外条件,才能使其具备强对偶性。

Slater's condition\footnote{https://zhuanlan.zhihu.com/p/58064316}是一个比较容易验证的强对偶性条件。
如果$\exists \mathrm{x}\in\mathrm{relint}\mathcal{D}$ ,使得$f_i(\mathrm{x})<0,i=1,...,m)$,那么就满足Slater's条件。这里$\mathrm{relint}\mathcal{D}=\{\mathrm{x}\in\mathcal{D}|B(\mathrm{x},r)\cap \mathrm{aff}\mathcal{D}\subseteq\mathcal{D},\ for\ some\ r>0\}$,也就是说存在一个点$\mathrm{x}$,在其周围的球$B(\mathrm{x},r)$与$\mathrm{aff}\mathcal{D}$相交的范围,是$\mathcal{D}$的子集,且这个点满足所有不等式约束。

\paragraph{互补-充实条件}
对于不等式约束$\lambda_i^\star,f_i(\mathrm{x}^\star)$,满足互补充实条件,意味着
$$
\lambda_i^\star f_i(\mathrm{x}^\star)=0,i=1,...,m
$$
也就是说,当$\lambda>0$的时候,要有$f(x)=0$;如果$f(x)<0$,就要求$\lambda=0$\key{这里需要更加详细的说明。}

\section{KKT条件}
假设函数$f_0,f_1,...,f_m,h_1,...h_p$都是可微分的。且$\mathrm{x}^\star$是原问题的最优解,$(\lambda^\star,\nu^\star)$是对偶问题的最优解,且两个问题的gap为0。

由于$\mathrm{x}$可以最小化$L(\mathrm{x},\lambda^\star,\nu^\star)$,就有
$$
\nabla f_0(\mathrm{x}^\star)+\sum_{i=1}^m\lambda_i^\star\nabla f_i(\mathrm{x}^\star)+\sum_{j=1}^p\nu^\star_j\nabla h_j(\mathrm{x}^\star)=0
$$

这条加上上述提及的各种约束,最终得到\key{KKT系统}:
\begin{equation}
    \label{equ9}
    \left\{
        \begin{aligned}
            &f_i(\mathrm{x}^\star)\le 0,&i=1...m\\
            &h_j(\mathrm{x}^\star)=0,&j=1...p\\
            &\lambda_i^\star\ge0,&i=1...m\\
            &\lambda_i^\star f_i(\mathrm{x}^\star)=0,&i=1...m\\
            &\nabla f_0(\mathrm{x}^\star)+\sum_{i=1}^m\lambda_i^\star\nabla f_i(\mathrm{x}^\star)+\sum_{j=1}^p\nu^\star_j\nabla h_j(\mathrm{x}^\star)=0
        \end{aligned}
        \right.
\end{equation}

这五个条件中,前两个是原本携带的等式不等式约束;三四是互补充实条件;五是KT条件。

\key{后续的算法围绕着KKT系统出发。}

\end{document}