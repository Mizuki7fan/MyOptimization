\documentclass[a4paper]{D:/repositories/MyDGP/latex/PaperReadingLog}
\usepackage{caption}
\usepackage{overpic}
\usepackage{float}

\usepackage{enumitem} 
\usepackage{amsfonts,amssymb,amsmath}

\usepackage{algorithm}%%算法伪代码,支持中文
\usepackage{algorithmic}
\setmainfont{Arial}

\begin{document}

\PaperInfo
{3.二次规划}
{朱}
{天宇}
{}

\section{二次规划}
二次规划是指等式、不等式约束都是二次的,所求的目标函数也是二次的情景。
$$
\begin{aligned}
    \min\quad Q(\mathrm{x})&=\frac{1}{2}\mathrm{x}^\top\mathrm{G}\mathrm{x}+\mathrm{c}^\top\mathrm{x}\\
    s.t.\quad \mathrm{a}_i^\top\mathrm{x}&=b_i,\ i\in\mathcal{E}=\{1,2,...m_e\}\\
    \mathrm{a}_i^\top&\ge b_i,\ i\in\mathcal{I}=\{m_{e+1},...m\}
\end{aligned}
$$

这里要求$\mathrm{G}$是对称阵、$\mathrm{a}_i(i\in\mathcal{E})$是线性无关的。

\subsection{分类}
\begin{enumerate}
    \item 如果约束是不相容的,则可能没有有限的最小值, 此时二次规划无解;
    \item 如果矩阵$\mathrm{G}$是半正定的,则问题是一个凸的二次规划问题,其任意局部解也是全局解;
    \item 如果矩阵$\mathrm{G}$是正定的,则问题是一个正定二次规划问题,只要存在解即是唯一解。
    \item 如果矩阵$\mathrm{G}$不定,则问题是一般的二次规划问题,有可能有非整体的局部解。
\end{enumerate}

\paragraph{正定矩阵含义}如果一个$n$阶方阵$\mathrm{M}$,对于任意非零向量$\mathrm{z}$,都有$\mathrm{z}^\top\mathrm{M}\mathrm{z}>0$,则称$\mathrm{G}$为正定矩阵。\key{正定矩阵所有特征值都为正数}。

\paragraph{半正定矩阵含义}如果$\mathrm{z}^\top\mathrm{M}\mathrm{z}\ge0$,则称$\mathrm{G}$为半正定矩阵。半正定矩阵所有特征值\key{非负}。

\paragraph{熟悉上方较为抽象的写法}设$
\mathrm{x}=\left(\begin{array}[]{c}
    x_1\\x_2\\x_3
\end{array}
\right)
$,$
\mathrm{c}=\left(\begin{array}[]{c}
    1\\2\\3
\end{array}
\right)
$

则根据$\mathrm{G}$的正定、半正定、不定的情况,分别举一个展开的例子。

$\mathrm{G}=\left(\begin{array}[]{ccc}
        2 & -1 & 0 \\
        -1 & 2 & -1 \\
        0 & -1 & 2 \\
    \end{array}
    \right)
    $是正定矩阵,因为$\mathrm{x}^\top\mathrm{G}\mathrm{x}=x_1^2+(x_1-x_2)^2+(x_2-x_3)^2+x_3^2>0$。此时原函数$\mathrm{x}^\top\mathrm{G}\mathrm{x}+\mathrm{c}^\top\mathrm{x}$展开得到

    $$
    Q=x_1^2+(x_1-x_2)^2+(x_2-x_3)^2+x_3^2+x_1+2x_2+3x_3
    $$

    对$x_1,x_2,x_3$分别求偏导,就可以解出唯一的值,因此只要存在解就是唯一解。

    $\mathrm{G}=\left(\begin{array}[]{ccc}
        1 & 0 & 0 \\
        0 & 2 & 0 \\
        0 & 0 & 0 \\
    \end{array}
    \right)
    $是半正定矩阵,此时原函数$\mathrm{x}^\top\mathrm{G}\mathrm{x}+\mathrm{c}^\top\mathrm{x}$展开得到

    $$
    Q=x_1^2+2x_2^2+x_1+2x_2+3x_3
    $$

    对$x_1,x_2,x_3$分别求偏导,其中$x_1,x_2$是有唯一解的,而$x_3$无解。原函数没有全局最小值($x_3\leftarrow-\infty$)

    当矩阵不定的时候,整体解(无约束的情况下)不一定存在。

\section{仅含等式约束的二次规划}
$$
\begin{aligned}
    \min\quad Q(\mathrm{x})&=\frac{1}{2}\mathrm{x}^\top\mathrm{G}\mathrm{x}+\mathrm{c}^\top\mathrm{x}\\
    s.t.\quad \mathrm{A}\mathrm{x}&=\mathrm{b}
\end{aligned}
$$

在这里,$\mathrm{A}$是一个$m\times n,m<n$的矩阵。不失一般性,设$rank(\mathrm{A})=m$,此时$\mathrm{A}$是行满秩的,因此可行域非空。

\subsection{基于消去法的两种方法}
\subsection{Lagrange方法}
拉格朗日方法基于求解可行域内的(K-T)点,即拉格朗日函数的稳定点。拉格朗日函数的稳定点是如下线性方程组的解:
$$\left\{
    \begin{aligned}
        \mathrm{G}\mathrm{x}+\mathrm{c}&=\mathrm{A}^\top\lambda\\
        \mathrm{A}\mathrm{x}&=\mathrm{b}
    \end{aligned}
    \right.
$$

写成矩阵形式:
$$
\left(
    \begin{array}[]{cc}
        \mathrm{G} & -\mathrm{A}^\top \\
        -\mathrm{A} & 0        
    \end{array}
\right)
\left(
    \begin{array}[]{c}
        \mathrm{x} \\ \lambda
    \end{array}
\right)
=
-\left(\begin{array}[]{c}
    \mathrm{c} \\ \mathrm{b}
\end{array}
\right)
$$

这是一个更高维的线性方程组,可以强行求解但是没有必要。基于左侧矩阵的特殊结构,可以通过矩阵分解的方式来获得解。

设$
\left(
    \begin{array}[]{cc}
        \mathrm{G} & -\mathrm{A}^\top \\
        -\mathrm{A} & 0        
    \end{array}
\right)$的逆矩阵为$
\left(
    \begin{array}[]{cc}
        \mathrm{U} & \mathrm{W}^\top \\
        \mathrm{W} & \mathrm{V}        
    \end{array}
\right)$

从而可以获得问题的唯一解为
$$
\left\{\begin{aligned}
    \mathrm{x}^\star&=-\mathrm{U}\mathrm{c}-\mathrm{W}^\top\mathrm{b}\\
    \lambda^\star&=-\mathrm{W}\mathrm{c}-\mathrm{V}\mathrm{b}
\end{aligned}
\right.
$$

从而可以计算出$\mathrm{U},\mathrm{V},\mathrm{W}$的表示,如图。但这样的表示需要计算$\mathrm{G}^{-1}$,这个计算代价和直接计算增广的线性方程组没啥区别。因此要用\key{矩阵分解}的方法。

\begin{figure}[H]%
    \centering
    \begin{overpic}[width=0.5\linewidth]{fig/3.1.png}
    \end{overpic}
    \caption{可以,但没必要。}
    \vspace{-3.5mm}
    \vspace{2mm}
\end{figure}

取$\mathrm{Y},\mathrm{Z}$满足$(\mathrm{Y},\mathrm{Z})=\left(\begin{array}[]{c}\mathrm{A}\\\mathrm{B}    
\end{array}
\right)^{-1}$,其中$\mathrm{Y}$是矩阵\key{商空间}的一组基构成的矩阵;$\mathrm{Z}$是解空间的基构成的矩阵;即$\mathrm{A}\mathrm{Y}=\mathrm{I}_{m\times m},\mathrm{A}\mathrm{Z}=0$。这里$\mathrm{A}$是$m\times n$的,$\mathrm{Y}$是$n\times m$的。此时有
$$
\left\{\begin{aligned}
    \mathrm{U}&=\mathrm{Z}(\mathrm{Z}^\top\mathrm{G}\mathrm{Z})^{-1}\mathrm{Z}^\top\\
    \mathrm{V}&=-\mathrm{Y}^\top\mathrm{G}\mathrm{P}^\top\mathrm{Y}\\
    \mathrm{W}&=-\mathrm{Y}^\top\mathrm{P}
\end{aligned}
    \right.
$$

这里$\mathrm{P}=\mathrm{I}-\mathrm{G}\mathrm{Z}(\mathrm{Z}^\top\mathrm{G}\mathrm{Z})^{-1}\mathrm{Z}^\top$。

这里$\mathrm{B}$取值我们不关心。所以就需要\key{根据矩阵$\mathrm{A}_{m\times n}$,确定$\mathrm{Y}_{n\times m},\mathrm{Z}$使得$\mathrm{A}\mathrm{Y}=\mathrm{I}$以及$\mathrm{A}\mathrm{Z}=0$}。

通过对$\mathrm{A}$做\key{QR分解}可以获得一组特殊的$\mathrm{Y},\mathrm{Z}$取值。

$Q$是一个$n\times n$的正交阵;$R$是一个$m\times m$的上三角阵;

设置$\mathrm{A}^\top=\mathrm{Q}\left(\begin{array}[]{c}
    \mathrm{R} \\ 0
\end{array}
\right)=(\mathrm{Q}_1,\mathrm{Q}_2)\left(\begin{array}[]{c}
    \mathrm{R} \\ 0
\end{array}
\right)$,从而有
$$
\mathrm{A}=(\mathrm{R}^\top,0)\left(
    \begin{array}[]{c}
        \mathrm{Q}_1^\top\\\mathrm{Q}_2^\top
    \end{array}
\right)
$$

维度分别为$\mathrm{Q}_{n\times n},\mathrm{R}_{m\times m}$,维度不足则补0。

可以令$\mathrm{Y}=\mathrm{Q}_1\mathrm{R}^{-\top},\mathrm{Z}=\mathrm{Q}_2$。

从而求出$\mathrm{U},\mathrm{V},\mathrm{W}$,进而求出$\mathrm{x}$。

\section{包含不等式约束的二次规划}

$$
\begin{aligned}
    \min\quad Q(\mathrm{x})&=\frac{1}{2}\mathrm{x}^\top\mathrm{G}\mathrm{x}+\mathrm{c}^\top\mathrm{x}\\
    s.t.\quad \mathrm{a}_i^\top\mathrm{x}&=b_i,\ i\in\mathcal{E}=\{1,2,...m_e\}\\
    \mathrm{a}_i^\top&\ge b_i,\ i\in\mathcal{I}=\{m_{e+1},...m\}
\end{aligned}
$$

\paragraph{active set}根据函数的连续性,\key{直观上,不积极的不等式约束在解的附近不起作用,可以去掉不予考虑;而积极的不等式约束,由于在解处等号成立,故而可以用等式约束来替代不等式约束。}从而转化成等式约束。

需要设法判断积极约束和不积极约束。

\subsection{积极集基本定理}

定义记号$\mathcal{I}(\mathrm{x}^k)=\{i\in\mathcal{I}|\mathrm{a}_i^\top\mathrm{x}^k=b_i\}$。这个记号是依赖于$\mathrm{x}^k$的,表示在不等式约束构成的集合$\mathcal{I}$中,在$\mathrm{x}^k$点处,使得约束$\mathrm{a}_i^\top\mathrm{x}^k=b_i$等号成立的这些约束。称这些约束在$\mathrm{x}^k$点处是积极的。

\paragraph{定理} 设$\mathrm{x}^\star$是一般的二次规划问题(包含所有等式和不等式约束)的局部极小点。则$\mathrm{x}^\star$也必然是等式约束问题
$$
(EQ)\left\{\begin{aligned}
    \min\quad Q(\mathrm{x})&=\frac{1}{2}\mathrm{x}^\top \mathrm{G}\mathrm{x}+\mathrm{c}^\top \mathrm{x}\\
    s.t.\quad \mathrm{a}_i^\top\mathrm{x}&=b_i,\ i\in\mathcal{E}\cup\mathcal{I}(\mathrm{x}^\star)
\end{aligned}
\right.
$$

的局部极小点。这个公式中$i$表示第$i$个不等式约束。反之,如果$\mathrm{x}^\star$是一般问题的可行点,同时是问题$(EQ)$的K-T点,且相应的拉格朗日乘子满足$\lambda^\star_i\ge 0,i\in\mathcal{I}^\star$,那么$\mathrm{x}^\star$必然是原问题的K-T点。

也就是说问题$(EQ)$的K-T点同时还是原问题的可行点,且不等式约束的拉格朗日乘子都大于等于0,那么$\mathrm{x}^\star$就是原问题的K-T点。

因此,记$\mathcal{E}_k=\mathcal{E}\cup\mathcal{I}(\mathrm{x}^k)$,尝试求解这个含等式约束的二次规划问题$EQ1$:
$$
(EQ1)\left\{\begin{aligned}
    \min\quad &\frac{1}{2}\mathrm{s}^\top \mathrm{G}\mathrm{s}+(\mathrm{G}\mathrm{x}^k+\mathrm{c})^\top \mathrm{s}\\
    s.t.\quad &\mathrm{a}_i^\top\mathrm{s}=0,\ i\in\mathcal{E}_k
\end{aligned}
\right.
$$

也就是使用\textbf{当前迭代点}$\mathrm{x}^k$对应的积极的不等式约束来替代局部极小点$\mathrm{x}^\star$对应的积极的不等式约束。

在这里,$\mathrm{s}=\mathrm{x}-\mathrm{x}^k$。结果的$\mathrm{s}$分为3种情况:
\begin{enumerate}
    \item $\mathrm{s}^k\neq 0$,此时$\mathrm{x}^k$不是原问题的K-T点;
    \item $\mathrm{s}^k=0$,则$\mathrm{x}^k$是问题
    $$
    (EQ2)\left\{\begin{aligned}
        \min\quad &\frac{1}{2}\mathrm{x}^\top \mathrm{G}\mathrm{x}+\mathrm{c}^\top \mathrm{x}\\
        s.t.\quad &\mathrm{a}_i^\top\mathrm{x}=b_i,\ i\in\mathcal{E}_k
    \end{aligned}
    \right.
    $$
    的K-T点。如果$\lambda_i^k\ge 0,i\in\mathcal{I}(\mathrm{x}^k)$,则$\mathrm{x}^k$也是原问题的K-T点。
    \item 否则,选择不等式条件$i_q$使得$\lambda_{i_q}^k=\min_{i\in\mathcal{I}(\mathrm{x}^k)}\lambda_i^k<0$,此时,解要满足的约束太多,需要释放掉一些。求如下问题
    $$
    (EQ3)\left\{\begin{aligned}
        \min\quad &\frac{1}{2}\mathrm{s}^\top \mathrm{G}\mathrm{s}+(\mathrm{G}\mathrm{x}^k+\mathrm{c})^\top \mathrm{s}\\
        s.t.\quad &\mathrm{a}_i^\top\mathrm{s}=0,\ i\in\mathcal{E}_k\verb|\|{i_q}
    \end{aligned}
    \right.
    $$
    的解$\hat{\mathrm{s}}$,作为当前点$\mathrm{x}^k$处的可行方向。即$\mathrm{a}_{i_q}^\top \hat{\mathrm{s}}\ge 0$。
\end{enumerate}


\subsection{二次规划的积极集方法}
\begin{algorithm}[H]
	\caption{积极集方法} 
	\begin{algorithmic}[1]
		% 初始化
		\STATE 初始化:给定可行点$\mathrm{x}^0$,令$\mathcal{E}_0\leftarrow\mathcal{E}\cup\mathcal{I}(\mathrm{x}^0),k\leftarrow 0$
        \WHILE{true}
            \STATE 求解等式约束问题(EQ1),获得下降方向$\mathrm{s}^k$
            \IF{$\mathrm{s}^k=0$}
                \IF{$\lambda_i^k\ge 0,i\in\mathcal{I}(\mathrm{x}^k)$}
                    \STATE \key{算法停止}
                \ELSE
                    \STATE 选择不等式约束$i_q$,满足$\lambda_{i_q}^k=\min_{i\in\mathcal{I}(\mathrm{x}^k)}\lambda_i^k<0$
                    \STATE $\mathcal{E}_k\leftarrow\mathcal{E}_k\verb|\|{i_q},\mathrm{x}^{k+1}\leftarrow \mathrm{x}^k$
                \ENDIF
            \ELSE
                \STATE 确定步长$\alpha_k\leftarrow\min\{1,\min_{i\in\mathcal{E}_k,\mathrm{a}_i^\top \mathrm{s}^k<0}\frac{b_i-\mathrm{a}^\top_i\mathrm{x}^k}{\mathrm{a}_i^\top\mathrm{s}^k}\}$
                \STATE $\mathrm{x}^{k+1}\leftarrow \mathrm{x}^k+\alpha_k\mathrm{s}^k$
                \IF[从不积极的约束中选择可以添加的积极约束]{$\alpha_k\neq 1$}
                    \STATE 寻找$p\notin\mathcal{E}_k$,满足$\mathrm{a}_p^\top(\mathrm{x}^k+\alpha_k\mathrm{s}^k)=b_p$
                    \STATE $\mathcal{E}_k\leftarrow \mathcal{E}_k\cup\{p\}$
                \ENDIF
            \ENDIF
            \STATE $\mathcal{E}_{k+1}\leftarrow \mathcal{E}_k,k\leftarrow k+1$
        \ENDWHILE
	\end{algorithmic}
\end{algorithm}

第14行是考虑不在非积极约束,决定步长的时候要考虑这些约束。步长移动可能导致原本的不积极约束变成了积极约束,也有可能使得可行解变成不可行解。通过等式约束决定方向、通过不等式约束决定步长,这是在多个问题中会用到的思想。

\subsection{二次规划的基于凸优化的方法}
如果满足$\mathrm{G}$是半正定的,则可以根据凸优化的相关方法来获取二次规划问题的解。

\end{document}